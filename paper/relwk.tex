\section{Related work}

\paragraph{Parsing frameworks.}

Generating parsers and generators from an executable specification
is the core concept of Interface Generators, e.g. in CORBA or more
recently~\cite{varda2008}. However, while interface generators work
very well for existing grammars, they do not allow full control over the
format of the output, so cannot be used to implement legacy protocols.
Very related work has been done at Bell Labs with the PacketTypes
system~\cite{mccann2000packet}, however PacketTypes works only as a
parser, not as an output generator and does not support the expressive
power of PEGs, but rather implements a C like structure model enhanced
with data-dependent length fields and constraints.

\paragraph{Language safety.}

Input processing vulnerabilities fall into two broad classes.  The first
class is memory safety bugs, such as buffer overflows, which allow an
adversary to corrupt the application's memory using specially crafted
inputs.  These mistakes arise in lower-level languages that do not
provide memory safety guarantees (such as C), and can be partially
mitigated by a wide range of techniques, such as static analysis,
dynamic instrumentation, and address space layout randomization, that
make it more difficult for an adversary to exploit these bugs.

The second class is logic errors, where application code misinterprets
input data.  This can lead to serious security consequences when two
systems disagree on the meaning of a network packet or a signed message,
as in iOS\footnote{The XNU kernel and the user code-signing verifier
interpret executable metadata differently, so the code signature
sees different bytes at a virtual address than the executable that
runs.}~\cite{evaders6} and Android\footnote{Android applications
are distributed as .zip files. Signatures are verified with a Java
program, but the program is extracted with a C program.  The java
program interprets all fields as signed, whereas the C program
treats them as unsigned, allowing one to replace files in a signed
archive.}~\cite{saurik-masterkey} code signing and even the X.509 protocol
underlying SSL~\cite{DBLP:conf/fc/KaminskyPS10}.  These mistakes are
highly application-specific, and are difficult to mitigate using existing
techniques, and these mistakes can occur even in high-level languages
that guarantee memory safety.


--- XXX integrate the following paragraphs somewhere into the logical flow ---

With currently prevalent software design methods, it is hard to isolate
input-handling code from the rest of the program, as user input is
typically passed through the program and processed in bits and pieces as a
``shotgun parser''~\cite{shotgun-parser}.

Many protocol implementations are so complicated that even
without bugs, they inadvertently provide a full Turing-complete execution
environment. Examples include X86 page tables~\cite{bangert2013page},
ELF symbols and relocations~\cite{shapiro2013weird}. Even though
these protocols were never intended to be that expressive -
and consequentially hard to parse, implementation details of their
recognisers - which de facto became a part of the specification and were
relied upon by legitimate inputs - expose unwanted computational power
and have unintended side-effects.  In the offensive research community,
this has been generalised into treating a program as a \textit{weird
machine}~\cite{bratus2011exploit} that operates on an input, analogous
to a virtual machine operating on bytecode.

Parser generators can automatically create parsers for such languages,
and with common classes of grammars such as regular expressions or
deterministic context free languages, that parser can be expressed in a
computationally less powerful form, e.g. a finite automaton or a push
down automaton, limiting the weird machine problem and enabling very
efficient implementation~\cite{Knuth1965607}.

Proper input recognition has been shown to be an excellent way of
eliminating malicious inputs. In one case, a  PDF parser implemented in
Coq could eliminate over $95\%$ of known malicious PDFs~\cite{Bogk-PDF}.
However, manually writing parser code does not scale to the number of
file formats  and protocols in existence and might result in parser code
tied to one specific application.

In fact, definitive grammars have long been written for all sorts of
protocols and are the standard way of describing text-based protocols.
Specifications of binary protocols, such as RFCs, also contain grammars,
although they are frequently translated into prose and diagrams as
opposed to the traditional Backus-Naur-Form.

