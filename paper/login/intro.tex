\begin{quote}
  I am ZIP, file of files.
  Parse me, ye mighty and drop a shell.

  -- Edward Shelley on the Android Master Key
\end{quote}
\section{Introduction}
Binary file formats and network protocols are hard to parse safely: 
The libpng image decompression library had 24 remotely
exploitable vulnerabilities from 2007 to 2013 according to CVEDetails,
Adobe's PDF and Flash viewers have been notoriously plagued by input
processing vulnerabilities, and even the zlib compression library had
input processing vulnerabilities in the past.
Most of these attacks involve memory corruption - therefore, it is easy to assume that 
solving memory corruption will end all our woes when handling untrusted inputs. 

However, as memory-safe languages and exploit mitigation  tricks are becoming more prevalent,
attackers are moving to a new class of attack - parser differentials. Many applications use
hand-written input processing code, which is often mixed with the rest of the application, e.g. by
passing a pointer to the raw input through the application. This (anti-)pattern makes it impossible
to figure out whether two implementations of the same format or protocol are identical, and input
handling code can't be easily reused between applications. As a result, different applications often
disagree in the corner cases of a protocol, which can have fatal security consequences. For example,
Android has two parsers for ZIP archives involved in securely installing applications. First, a Java
program checks the signatures of files contained within an app archive and then another tool
extracts them to the filesystem. Because the two ZIP parsers disagree in multiple places, attackers
can modify a valid file so that the verifier will see the original contents, but attacker-controlled
files will be extracted to the filesystem, bypassing Android's code signing. Similar issues showed
up on iOS \cite{geohot-evasion} and SSL \cite{DBLP:conf/fc/KaminskyPS10}. 

Instead of attempting to parse inputs by hand and failing, 
a promising approach is to specify
a precise grammar for the input data format, and automatically generate parsing code from that with tools like {\tt yacc}. As long
as the parser generator is bug-free, the application will be safe from many
input processing vulnerabilities. Grammars can also be re-used between
applications, further reducing effort and eliminating inconsistencies.

This approach is typical in compiler design and in other applications handling text-based inputs, but not
common for binary inputs. The Hammer framework~\cite{hammer-parser} and data description languages such as
PADS~\cite{Fisher:2005:PDL:1064978.1065046} have been developing generated parsers for binary protocols.

However, if you wanted to use existing tools to parse PDF
or ZIP, you would soon find that they cannot handle the complicated -- and therefore most
error-prone-- aspects of such formats-- so you'd still have to
hand-write the riskiest bits of code. For example, existing parser generators cannot conveniently represent size or offset fields and more complex features, such as data compression or checksums cannot be expressed at all.

Furthermore, some parser generators are cumbersome to use when parsing binary data for several
reasons. First, many parser generators don't produce convenient data structures, but call
\emph{semantic actions} that you have to write to build up a data structure your program can use. Therefore,you must
describe the format up to three times -- in the grammar, the data structure and the semantic
actions. Second, most parser generators only address parsing inputs, so you have to manually
construct outputs. Some parser generators, such as 
Boost.Spirit  allow generating output, but require you to write another set of semantic actions.


We address these challenges with Nail, a new parser generator for binary formats.
First, Nail grammars describe not only a format, but also a data type to represent it within the program.
Therefore you don't have to write semantic actions and type declarations and you can no longer syntactic validation and semantic processing.
Second, Nail will also generate output from this data type, without requiring you
to write more risky code or giving you a chance to introduce ambiguity.

Third,  Nail introduces two abstractions, \emph{dependent fields} and
\emph{transformations}, to elegantly handle problematic structures,
such as offset fields or checksums.  Dependent fields capture fields
in a protocol whose value depends in some way on the value or layout
of other parts of the format; for example, offset or length fields,
which specify the position or length of another data structure, fall
in this category.  Transformations allow you to write plug-ins, allowing your programs to handle complicated structures, while
keeping Nail itself small, yet flexible. 

In the rest of this article, we will show some tricky features of real world formats and how to
handle them with Nail.

%% To evaluate whether Nail's design is effective at handling real-world data
%% formats, we implemented a prototype of Nail for C\@. Using our prototype, we
%% implemented grammars for parts of an IP network stack, for DNS packets,
%% and for ZIP files, each in about a hundred lines of grammar. On top of these
%% grammars, we were able to build a DNS server in under 200 lines of C code, and
%% an {\tt unzip} utility in about 50 lines of C code, with performance comparable
%% to or exceeding existing implementations. This suggests both that Nail is
%% effective at handling complex real-world data formats, and that Nail makes it
%% easy for application developers to parse and generate external data
%% representations. Performance results show that the Nail-based DNS server
%% outperforms the widely used BIND DNS server, demonstrating that Nail-based
%% parsers and generators can achieve good performance.

%% The rest of this paper is organized as follows.  \S\ref{s:relwk}
%% puts Nail in the context of related work.
%% \S\ref{s:motivation} motivates the need for Nail by examining
%% past data format vulnerabilities.
%% \S\ref{s:design} describes
%% Nail's design.  \S\ref{s:impl} discusses our implementation
%% of Nail.  \S\ref{s:eval} provides evaluation results, and
%% \S\ref{s:concl} concludes.

