\section{Evaluation}
\label{s:eval}

% In our evaluation of Nail, we answer four questions:

% \begin{itemize}

% \item Can Nail grammars support real-world data formats, and
%       are Nail's techniques critical to handling these formats?

% \item How much programmer effort is required to build an
%       application that uses Nail for data input and output?

% \item Does using Nail for handling input and output improve
%       application security?

% \item Does Nail achieve acceptable performance?

% \end{itemize}

\subsection{Real-World Formats}
\label{s:eval-formats}

First, we ask whether Nail can handle real-world data formats and whether its techniques are
critical to that. We used Nail to implement grammars
for seven protocols with a range of challenging features.
Figure~\ref{fig:eval-protocols} summarizes our results.
We find that despite the challenging aspects of these protocols, Nail is
able to capture the protocols, by relying on its novel features: dependent
fields, streams, and transforms.  In contrast, state-of-the-art parser
generators would be unable to fully handle 5 out of the 7 data formats.
In the rest of this subsection, we describe the DNS and Zip grammars in
more detail, focusing on how Nail's features enable us to support these
formats.

\begin{figure}[tb]
\centering
\begin{tabular}{lrl}
\toprule
\textbf{Protocol} & \textbf{LoC} & \textbf{Challenging features} \\ 
\midrule
DNS packets & 48+64 & Label compression,\\
  & & count fields \\
ZIP archives & 92+78 & Checksums, offsets, \\ 
  & & variable length trailer, \\
  & & compression \\
Ethernet  & 16+0\phantom{0} & --- \\
ARP       & 10+0\phantom{0} & --- \\
IP        & 25+0\phantom{0} & Total length field, options \\
UDP       &  7+0\phantom{0} & Checksum, length field \\
ICMP      &  5+0\phantom{0} & Checksum \\
\bottomrule
\end{tabular}

\caption{Protocols, sizes of their Nail grammars, and challenging aspects
of the protocol that cannot be expressed in existing grammar languages.
A + symbol counts lines of Nail grammar code (before the +) and lines of
C code for protocol-specific transforms (after the +).}
\label{fig:eval-protocols}
\end{figure}

\paragraph{DNS} In Section~\ref{s:design}, we introduced Nail's syntax with a grammar
for DNS packets, shown in Figure~\ref{fig:dns-full}.
The grammar corresponds almost directly to the diagrams in RFC 1035,
which defines DNS~\cite[\S4]{RFC:1035}. Nail's dependent fields handle DNS's count fields, and
transformations represent label compression. Both of these features are awkward at best to handle
with existing tools.


\paragraph{ZIP} An especially tricky data format is the ZIP compressed archive format~\cite{pkzip}.
ZIP files are normally parsed end-to-beginning. At the end of each ZIP file is an
\emph{end-of-directory header}. This header contains a variable-length comment, so it has to be
located by scanning backwards from the end of the file until a magic number and a valid length field
is found. Many ZIP implementations disagree on the exact semantics of this, such as when the comment
contains the magic number~\cite{wolf:berlinsides-zip}.

This header  contains points to the \emph{ZIP directory}, contains an entry every file
in the archive.  Each entry stores file metadata in addition to the
offset of a \emph{local file header}. The local file header duplicates
most information from the directory entry header and is followed immediately by the compressed
archive entry.
Duplicating information made sense when ZIP files were stored on floppy
disks with slow seek times and high fault rates, and memory constraints
made it impossible to keep the ZIP directory in memory. 

 However, if attackers craft inconsistent files, hand written parsers fail in interesting ways, e.g. 
the length in the central directory is used to allocate memory and the
length in the local directory is used to extract without checking that
they are equal first, as was the case in the Python ZIP
library~\cite{python-bug:20078}. 

Nail avoids these issues by capturing these redundancies with dependent fields and decompressing
files transparently with transformations, and the programmer
is presented with a data structure containing the archive contents. To handle formats based on ZIP,
such as office documents, our ZIP grammar could be extended to apply additional grammars to the
files within an archive.



\subsection{Applications}
\label{s:eval-effort}

%code size unzip 6.0
%2821 lines
We implemented two applications---a
DNS server and an
\cc{unzip} program---based on the above grammars, and will compare the effort involved and the
resulting security to similar applications with hand-written parsers and with other parser generators.
We will use lines of code as a proxy for programmer effort. To evaluate security, we will argue how
our design avoids classes of vulnerabilities and fuzz-test one of our applications.
\begin{figure}[tb]
\centering
\smaller[0.5]
\begin{tabular}{lrr@{~}l}
\toprule
\textbf{Application}
  & \textbf{LoC w/ Nail}
  & \multicolumn{2}{c}{\textbf{LoC w/o Nail}} \\
\midrule
DNS server
  & 295
  & 683
  & (Hammer parser) \\

%  &
%  & 10,000
%  & (DJBDNS) \\

\cc{unzip}
  & 220
  & 1,600
  & (Info-Zip) \\
\bottomrule
\end{tabular}
\caption{Comparison of code size for three applications written in
  Nail,and a comparable existing implementation without Nail.}
\label{fig:effort}
\end{figure}

\paragraph{DNS.}

Our DNS server parses a zone file, listens to incoming DNS requests,
parses them, and generates appropriate responses.  The DNS server is
implemented in 183 lines of C, together with 48 lines of Nail grammar
and 64 lines of C code implementing stream transforms for DNS label
compression.  In comparison, Hammer~\cite{hammer-parser} ships with a toy
DNS server that responds to any valid DNS query with a CNAME record to the
domain ``spargelze.it''.  Their server consists of 683 lines of C, mostly
custom validators, semantic actions, and data structure definitions, with
 52 lines of code defining the grammar with Hammer's combinators.
Their DNS server does not implement label compression, zone files, etc.
From this, we conclude that Nail leads to much more compact code for
dealing with DNS packet formats.

We ran the DNS fuzzer provided with the Metasploit
framework~\cite{mspframework} on our DNS server, which sent randomly
corrupted DNS queries to our server for 4 hours, during which it did
not crash or trigger the stack or heap corruption detector.

To evaluate whether Nail-based parsers are compatible with good
performance, we compare the performance of our DNS server to that of ISC
BIND 9 release 9.9.5~\cite{bind9}, a mature and widely used DNS server.
We simulate a load resembling that of an authoritative name server, generating a random zone file
and a random sequence of queries, with 10\% non-existent domains. We repeated this sequence of
queries for one minute against both DNS servers. We found that our DNS server is approximately three
times faster than BIND. Although BIND is a more sophisticated DNS server,
and implements many features that are not present in our Nail-based DNS
server and that allow it to be used in more complicated configurations, 
we believe our results demonstrate that Nail's parsers are not
a barrier to achieving good performance.
\paragraph{ZIP.}

We implemented a ZIP file extractor in 50 lines of C code, together with
92 lines of Nail grammar and 78 lines of C code implementing two stream
transforms (one for the DEFLATE compression algorithm with the help of
the zlib library, and one for finding the end-of-directory header).
 The \cc{unzip} utility contains a file \cc{extract.c}, which parses
ZIP metadata and calls various decompression routines in other files. This
file measures over 1,600 lines of C, which suggests that Nail is highly
effective at reducing manual input parsing code, even for the complex
ZIP file format.

In our full paper~\cite{bangert:nail-osdi14}, we present a study of 15 ZIP parsing bugs.
11 of these vulnerabilities involved memory corruption during input handling, which generated code is immune to
memory corruption attacks by design.
More interestingly, Nail also protects against parsing inconsistency vulnerabilities like the four others we studied.
Nail grammars explicitly encode duplicated information such as the redundant length fields in ZIP
that caused a vulnerability in the Python ZIP library. The other three vulnerabilities exist
because multiple implementations of the same protocol disagree on some inputs. Hand-written protocol
parsers are not very reusable, as they build application-specific data structures and are tightly
coupled to the rest of the code. Nail grammars, however, can be re-used between applications,
avoiding protocol misunderstandings.



