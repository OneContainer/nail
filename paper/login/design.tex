\section{Design By Example}
In this section, we will explain how to handle basic data formats in Nail, how to handle
redundancies in the format with dependent fields and how Nail parsers can be extended with
transformations. 

As a motivating example, we will parse DNS packets, as defined in RFC 1035. Each DNS packet consists of a
header, a set of question records, and a set of answer records. Domain
names in both queries and answers are encoded as a sequence of labels,
terminated by a zero byte.  Labels are Pascal-style strings, consisting
of a length field followed by that many bytes comprising the label.
\label{s:design}
\subsection{Basic Data formats}

\begin{figure}
\smaller[0.5]
\begin{verbatim}
labels = <many { @length uint8 | 1..64
                 label n_of @length uint8 }
          uint8 = 0>
compressed_labels = {
  $decompressed transform dnscompress ($current)
  labels apply $decompressed labels
}
question = {
  labels compressed_labels
  qtype uint16 | 1..16
  qclass uint16 | [1,255]
}
answer = {
  labels compressed_labels
  rtype uint16 | 1..16
  class uint16 | [1]
  ttl uint32
  @rlength uint16
  rdata n_of @rlength uint8
}
dnspacket = {
  id uint16
  qr uint1
  opcode uint4
  aa uint1
  tc uint1
  rd uint1
  ra uint1
  uint3 = 0
  rcode uint4
  @qc uint16
  @ac uint16
  @ns uint16
  @ar uint16
  questions n_of @qc question
  responses n_of @ac answer
  authority n_of @ns answer
  additional n_of @ar answer
}
\end{verbatim}
\caption{Nail grammar for DNS packets, used by our prototype DNS server.}
\label{fig:dns-full}
\end{figure}

Let's step through a simplified Nail grammar for DNS
packets, shown in  Figure~\ref{fig:dns-full}.  For this grammar, Nail produces the type declarations
shown in Figure~\ref{fig:dns-full-struct}, and the parser and
generator functions shown in Figure~\ref{fig:dns-full-api}.
Nail grammars are reuseable between applications and we will use this grammar to implement
both a DNS server and client, which previously would have had two separate hand-written parsers,
leading to bugs such as  the Android Master Key. 

A Nail grammar file consists of rule definitions, for example l. 1-20,  which assign a name
(\texttt{dnspacket}) to a grammar production (2-20). If you are not familiar with
other parsers, you can imagine rules as C type declarations on steroids
(although our syntax is inspired by Go).

Just like C supports various constructs to build up types, such as structures and unions from pointers and
elemental types, Nail supports various \emph{combinators} to represent features of a file or protocol. We will present the features we used in implementing
DNS. A more complete reference can be found in~\ref{fig:syntax}, with detailed rationale in~\cite{bangert:nail-osdi14}.



\paragraph{Integers and Constraints}.
Because Nail is designed to cope with binary formats,
it handles not only common integer types (such as \texttt{uint16} on l. 16), but bit fields of any
length, such as \texttt{uint1}. These integers are exposed to the programmer as an appropriately
sized machine integer, e.g. \texttt{uint8\_t}. Nail also supports constraints on integer values,
limiting the values to either a range (l. 23, \texttt{|1..16}), which can optionally be half-open,
or a fixed set (l. 24, \texttt{|[1,255]}). Both types of constraint can be combined, e.g. \texttt{|
  [1..16,255]}. Constant values are also supported, e.g. l. 10: \texttt{uint3=0} represents
three reserved bits that must be 0. Because constant values carry no information, they are not
represented in the data type.

\paragraph{Structures.}

The body of the \texttt{dnspacket} rule is a structure, which contains any number of fields enclosed
between  curly braces. Each field in the structure is parsed in sequence, and represented as a
structure to the programmer.  Contrary to other programming languages, Nail does not have a special
keyword for structs. We also reverse the usual structure-field syntax: \texttt{id uint1} is a field
called \texttt{id} with type \texttt{uint1}. Often, Nail grammars have
structures with just one non-constant field, for example when parsing a fixed header. Nail supports
this with an alternative form of structures,
using angle brackets, that contains one unnamed, non-constant field, which is represented directly
in the datatype, without introducing another layer of indirection, as shown on line 40.


\paragraph{Arrays.}

Nail supports various forms of arrays. Line 40 shows how to parse a domain in a DNS packet with
\cc{many}, which keeps repeating the \cc{label} rule until it fails.  In the next section, we will
explain how to handle count fields, and our full paper describes how to handle various array
representations (such as delimiters or non-empty arrays).


% To support recursive structures, we include a reference combinator \cc{*}, which
%   is syntactically equivalent to directly including the parser, but in the internal model adds a
%   level of indirection to the value. This allows recursive structures to be parsed, f

\begin{figure*}
\begin{tabular}{@{}p{5cm}p{6cm}p{5cm}@{}}
\toprule
\bf Nail grammar & \bf External format & \bf Internal data type in C\\
\midrule
\cc{uint4} & 4-bit unsigned integer.& \lstinline$uint8_t$ \\\hline
\cc{int32 | [1,5..255,512]} & Signed 32 bit integer $x \in \{ 1, 512 \} \vee 5\leq x\leq 255$ &
\lstinline$int32_t$ \\\hline

\cc{uint8=0}& 8-bit constant with value 0.& \lstinline$//Empty$\\\hline

\cc{optional int8|16..} & 8-bit integer $\geq 16$ or nothing & \lstinline$int8_t *$\\\hline

\cc{many int8 | ![0]} & A NUL-terminated string. & 
\begin{minipage}{5cm}
\begin{lstlisting}
struct {
  size_t N_count;
  int8_t *elem;
};
\end{lstlisting}
\end{minipage}\\\hline

\begin{minipage}{5cm}
\begin{verbatim}
{ 
 hours uint8
 minutes uint8
}
\end{verbatim}
\end{minipage}
& Structure with two fields. &
\begin{minipage}{5cm}
\begin{lstlisting}
struct {
  uint8_t hours; 
  uint8_t minutes;
};
\end{lstlisting}
\end{minipage}
\\\hline

\begin{minipage}{5cm}
\verb+<int8 = '"'+\\
A parser definition \textit{p}\\
\verb+int8='"'>+
\end{minipage}
\cc{} & Lowercase string enclosed in quotes. & The data type of {\it p} \\\hline


\begin{minipage}{5cm}
\begin{verbatim}
choose {
  A = uint8 | 1..8
  B = uint16 | 256..
}
\end{verbatim}
\end{minipage}
&\begin{minipage}{6cm} Either an 8-bit integer between 1 and 8 or a 16-bit integer larger than $256$.
  \end{minipage}&
\begin{minipage}{5cm}
\begin{lstlisting}
struct{
  enum {A,B} N_type;
  union {
    uint8_t a;
    uint16_t b;
  };
};
\end{lstlisting}
\end{minipage}\\\hline

\begin{minipage}{5cm}
\begin{verbatim}
{ 
 @valuelen uint16
 value n_of @valuelen uint8
 }
\end{verbatim}
\end{minipage}
&
\begin{minipage}{6cm}
A 16-bit field, followed by a length field, followed by that many bytes.
\end{minipage}
&
\begin{minipage}{5cm}
\begin{lstlisting}
struct{ struct{ 
    size_t N_count;
    uint8_t *elem;
} value;}
\end{lstlisting}
\end{minipage}
\\\hline
\begin{minipage}{5cm}
\vspace{0.5em}
\begin{verbatim}
data transform deflate(
    $current @method)
\end{verbatim}
\end{minipage}
&
Applies a programmer-specified function to create a new stream based on the current input stream and
a dependent field. See
Section~\ref{s:transforms}.&\lstinline+//none+ \\\hline
\verb+apply $stream many +\textsl{p}\textit{/*parser*/}&  Parse $p$ from a different stream, then
resume parsing the current stream& Data type of $p$ \\
\bottomrule
\end{tabular}
\caption{Syntax of Nail parser declarations and the formats and data types they describe.}
\label{fig:syntax}
\end{figure*}

\subsection{Redundant Data}
\label{s:dependent}
Data formats often contain values that are determined by other values or the layout of information,
such as checksums, duplicated information, or offset and  length fields. Exposing such values risks
inconsistencies that could trick the program into unsafe behaviour. Therefore, we represent such values using \emph{dependent fields} and handle them transparently during
parsing and generation without exposing them to the application. 
Dependent fields are handled like other fields when parsing input, but only stored temporarily
instead of in the data type. Their value can be referenced by other parsers until it goes out of scope.
When generating output, Nail inserts the correct value.

In DNS packets, the packet header contains count fields (\cc{qc},
\cc{ac}, \cc{ns}, and \cc{ar}), which contain the number of questions and answers that follow the
header, which we represent by dependent fields (line 12-15).
Dependent fields are defined within a structure like normal fields, but their name starts with an \cc{@} symbol.
A dependent field is in scope and can be referred to by the definition of all subsequent fields in
the same structure. Dependent fields can be passed to rule invocations as parameters.


Nail allows handling count fields with \cc{n\_of}, which parses an exact number of repetitions of a rule. Lines 16-19 in Figure~\ref{fig:dns-full} shows how to use \cc{n_of} to parse the question and
answer records in a DNS packet.
Other dependencies, such as offset fields or checksums, are not handled directly by combinators, but
through  transformations, as we describe next.

\subsection{Input streams and transformations}
\label{s:transforms}
So far, we have described a parser that consumes input a byte at a time from beginning to end.
However, real-world formats often require non-linear parsing. Offset fields require a parser to move
to a different position in the input, possibly backwards. Size fields require the parser to stop
processing before the end of input has been reached.
Other cases, such as compressed data and checksums, require more complicated processing on parts of the input
before it can be handled.


For a parser to be useful, it needs to support all these ways of structuring a format. This makes data description languages like
PADS~\cite{Fisher:2005:PDL:1064978.1065046} contain not just a kitchen sink, but a kitchen store
full of features -- and a language that can handle all possible formats will be a general
purpose programming language.  Instead, we keep Nail itself small, and introduce an
interface that allows complicated format structures to be handled by  plug-in \emph{transformations} 
in a general purpose
language. Of course, we ship Nail with a handy library of common transformations to handle common
format features, such as offsets,sizes and checksums.

These \emph{transformations} consume and produce
\emph{streams} -- sequences of bytes -- which can be further passed to other transformations and
eventually parsed by a Nail rule. Transformations can also access values in dependent fields. 
Streams can be subsets of other streams, for example the substream starting at an
offset given in a dependent field to handle pointer fields, or computed at runtime
, such as by decompressing another stream with zlib.


Transformations are two arbitrary functions called during parsing and output generation.
The parsing function consumes any number of streams and dependent field values,
and produces any number of temporary streams. This function may reposition and read from the
input streams and read the values of dependent fields, but not change their contents and values. 
The generating function has to be an inverse of the parsing function, consuming streams and
producing dependent field values and other streams.

\begin{figure}
\begin{verbatim}
!LITTLE-ENDIAN //Cut for conciseness
fileentry(@crc32 uint32,
 @method uint16,
 @compressed_size uint32, 
 @uncompressed_size uint32) = { 
  uint32 = 0x04034b50
  version uint16
  flags file_flags
  @method_local uint16//[...]
  $compressed transform size_u32 
           ($current @compressed_size)
  $uncompressed transform zip_compression 
           ($compressed @method )
  transform crc_32 ($uncompressed @crc32)
  contents apply $uncompressed many uint8
  transform u16_depend (@method_local @method)//[...]
}
dir_entry($file) = {
//[...]  
  @compression_method uint16      
  mtime uint16
  mdate uint16
  @crc32 uint32
  @compressed_size uint32
  @uncompressed_size uint32
  @file_name_len uint16
  @extra_len uint16
  @comment_len uint16//[...]
  @off uint32
  filename n_of @file_name_len uint8
  extra_field n_of @extra_len uint8
  comment n_of @comment_len uint8
  $content transform offset_u32 ($file @off)
  contents apply $content fileentry(@crc32,
     @compression_method,@compressed_size, 
     @uncompressed_size)
}
end_of_directory($file) = {//[...]
 @directory_size uint32 
 @directory_start uint32
 $dirstr1 transform offset_u32 
     ($filestream @directory_start) 
 $directory_stream transform size_u32 
     ($dirstr1 @directory_size)
 @comment_length uint16
 comment n_of @comment_length uint8
 files apply $directory_stream n_of 
     @total_directory_records dir_entry($file)
}
zip_file = { 
 $file, $directory transform 
    zip_end_of_directory ($current)
  contents apply $end_directory
    end_of_directory($file)
}
\end{verbatim}
\caption{Extracted Nail grammar for ZIP files.}
\label{fig:zip-extract}
\end{figure}


As a concrete example, we will show a grammar for ProtoZIP, a very simple archive format inspired by ZIP in
Figure~\ref{fig:zip-extract}. ProtoZIP consists of a variable-length end-of-file directory, which
is a magic number followed by an array of filenames and pointers to compressed files. A grammar for the real ZIP format, which
has more layers of indirection, is presented in the full paper. 

 The grammar first calls the zipdir transform on line 2, which finds the magic number and splits the file into
 two streams, one containing the compressed files, the other the directory.  Streams are referred to
 with \cc{\$identifiers}, similar to dependent fields. A C prototype of the \cc{zipdir} transform is
 shown in\ref{fig:zip-transform}. 
\begin{figure}[tb]
\smaller[0.5]
\begin{Verbatim}[commandchars=\\\{\},codes={\catcode`\$=3\catcode`\^=7\catcode`\_=8}]
\PY{k+kt}{int} \PY{n+nf}{zip\PYZus{}end\PYZus{}of\PYZus{}directory\PYZus{}parse}\PY{p}{(}
  \PY{n}{NailArena} \PY{o}{*}\PY{n}{tmp}\PY{p}{,} \PY{n}{NailStream} \PY{o}{*}\PY{n}{out\PYZus{}files}\PY{p}{,}
  \PY{n}{NailStream} \PY{o}{*}\PY{n}{out\PYZus{}dir}\PY{p}{,} \PY{n}{NailStream} \PY{o}{*}\PY{n}{in\PYZus{}current}\PY{p}{)}\PY{p}{;}
\PY{k+kt}{int} \PY{n+nf}{zip\PYZus{}end\PYZus{}of\PYZus{}directory\PYZus{}generate}\PY{p}{(}
  \PY{n}{NailArena} \PY{o}{*}\PY{n}{tmp}\PY{p}{,} \PY{n}{NailStream} \PY{o}{*}\PY{n}{in\PYZus{}files}\PY{p}{,}
  \PY{n}{NailStream} \PY{o}{*}\PY{n}{in\PYZus{}dir}\PY{p}{,} \PY{n}{NailStream} \PY{o}{*}\PY{n}{out\PYZus{}current}\PY{p}{)}\PY{p}{;}
\end{Verbatim}

\caption{Signatures of stream transform functions for handling the
end-to-beginning structure of ProtoZIP files.}
\label{fig:zip-transform}
\end{figure}
When parsing input, this will call \cc{zipdir\_parse}, which takes \cc{\$current} -- an implicit
identifier always refering to the stream currently being handled -- and returns \cc{\$files} and
\cc{\$header}. When generating output, this will call \cc{zipdir\_generate}, which appends
\cc{\$files} and \cc{\$header} to \cc{\$current}.


Line 3 then applies the \cc{dir} rule to  the \cc{\$header} stream, passing it the \cc{\$files}
stream. Within \cc{dir}, \cc{\$current} is now \cc{\$header} and input is parsed from and output
generated to that stream. The \cc{dir} rule in turn describes the structure of the directory -- a
magic number and a count field, followed by that many file descripors. Each file descriptor, is then parsed with two
transformations, the standard-library \cc{slice} which describes an offset and a size within another
stream, and the custom \cc{zlib}, which compresses a stream using zlib. Finally, we apply a trivial
grammar (line 14) to the contents. 

In a more complicated example, such as an Office document, we could now specify grammars for each
entry within an archive.


 

Transformations need to be carefully written, because they can violate Nail's safety properties
and introduce
bugs. However, as we will show in \S\ref{s:eval-effort}, Nail transformations are much shorter than
hand-written parsers, and many formats can be represented with just the transformations in Nail's
standard library.
For example, our Zip transformations are 78 lines of code, compared to 1600 lines of code for a
   hand-written parser. Additionally, Nail provides convenient and safe interfaces for allocating
   memory and accessing streams that address the most common occurrences of buffer overflow
   vulnerabilities.  


% For example, we imagine the following grammar could be used to represent
% a sequence of bytes \texttt{data} followed by its CRC32 checksum:

% \begin{verbatim}
% data many uint8; @checksum uint32
% raw_depend @checksum data crc32
% \end{verbatim}

% \noindent
% where \texttt{crc32} is a function supplied by the application, with
% the following signature:

% \begin{verbatim}
% bool crc32(uint32_t *out, uint8_t *in);
% \end{verbatim}

% Because this feature compromises Nail's security guarantees, it should
% only be used in limited circumstances and with carefully prepared checksum
% functions.  This feature is not implemented in the current prototype.

