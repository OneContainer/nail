\section{Implementation}
\label{s:impl}

The current prototype of the Nail parser generator supports the C programming
language and top down parsers. Options for C++ STL data models and emitting
Packrat parser~\cite{packrat-parsing:icfp02} are under heavy development. In
this section, we will discuss some particular features of our parser
implementation.

The source code of our implementation, together with the examples described in
\S\ref{s:eval} are available on GitHub\footnote{\url{https://github.com/jbangert/nail}}.
\paragraph{Parsing}
A generated Nail parser makes two passes through the input , the first to
validate and recognise the input, the second to bind this data to the internal
model. Currently the parser is a straightforward top down parser, although
facilities have been made to add Packrat parsing to achieve linear parsing
times.
\paragraph{Memory allocation}
Many security vulnerabilities can occur when heap allocations are done
improperly. Therefore, just like Hammer, Nail avoids using the heap as much as
possible, using a custom arena allocator and allocating only fixed size blocks
from the system malloc. However, Nail extends upon Hammers approach and uses two
arenas for each parsed input. One arena is used for intermediate results and is
freed(and zeroed) after parsing completes, whereas the other arena is used only
for allocating the result and has to be freed by the programmer.

\paragraph{Intermediate representation}
Most parser generators, such as Bison do not have to dynamically allocate
temporary data, as they evaluate a semantic action on every rule. However, as
our goal is to perform as little computation as possible without having
validated the input, and we do not want to mix temporary objects with the
results of our parse, we use an append-only trace to store intermediate parser
results. 

Hammer solves this problem by storing a full abstract syntax tree. However, this
abstract syntax tree is at least an order of magnitude larger than the input,
because it stores a large tree node structure for each input byte and for each
rule reduced. This allows Hammer semantic actions to get all necessary
information without ever seeing the raw input stream. However, because we also
automatically generate our second pass, which corresponds to Hammers semantic
actions, we can trust it as much as we trust the parser and thus expose it to
the raw input stream.

Under this premise, the actions need very limited information from the
recogniser to correctly handle the input stream. In particular, the parsers
control flow only branches at the choice, repetition and constant
combinators.
Thus, for each of those combinators, we store the minimum amount of information
required to reconstruct the syntactic structure of the input. The trace is an
array of integers.
 Whenever the parser encounters a choice, it appends two integers to the trace. After it
successfully parses a choice, the parser writes the number of that choice and
the length of the trace when it began parsing that choice. When backtracking in
the input, the parser does not backtrack in the trace. This means that offsets
within the trace can be used for a Packrat hash-table to memoise backtrack-heavy
parsers.

When encountering a repetition combinator, the parser records the number of
times the inner parser succeeded and when encountering a constant parser of
variable size, it records how much input was consumed by the constant parser. 

In a second pass, the parser then allocates the internal representation from an
arena allocator and binds the fields to values from the input, while following
the trace to determine how many array fields to parse and which choices to pick.
\paragraph{Dependency Fields}
During parsing, dependency are parsed  before the context in which they are
used. The parser stores their value and retrieves it afterwards when
encountering the combinator that uses them. When generating output, the
dependency field is first filled with a filler value, then later when the first
combinator that determines this fields value is encountered, the field is
overwritten. Any further combinators using this dependency will then validate
that the dependency field is correct.
\paragraph{Bootstrapping}
To demonstrate the feasibility of the nail parser generator, our parser
generator uses a Nail parser to recognise Nail grammars. A superset of the
grammar language described in this paper is implemented in 100 lines of Nail,
which feed into about a thousand lines of C++ that implement the actual parser
generator. Bootstrapping is supported via an implementation of the Nail language
in the Lemon parser generator, a variant of Yacc.
