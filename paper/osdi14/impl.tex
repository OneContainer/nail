\section{Implementation}
\label{s:impl}

The current prototype of the Nail parser generator supports the C programming
language. The implementation parses Nail grammars with Nail itself, using a
130-line Nail grammar feeding into a 2,000-line C++ program that
emits the parser and generator code. Bootstrapping is
performed with a subset of the grammar implemented using conventional grammars. 
An option for C++ STL data models is in development. In
this section, we will discuss some particular features of our parser
implementation.

The source code of our implementation, together with the examples described in
\S\ref{s:eval} are available on GitHub at \url{https://github.com/jbangert/nail}.


\paragraph{Parsing.}

A generated Nail parser makes two passes through the input: the first to
validate and recognize the input, and the second to bind this data to the internal
model. Currently the parser uses a straightforward top-down algorithm, which can perform poorly on
grammars that backtrack heavily. However, preparations have been made to add Packrat
parsing~\cite{packrat-parsing:icfp02} that achieve linear time even in the worst case.

\XXX[Add API here]
\paragraph{Defense-in-Depth.}

Security exploits often rely on raw inputs being present in memory~\cite{shotgun-parser}, for example to include
shell-code or crafted stack frames for ROP~\cite{phrack58:4-nergal}-attacks in padding fields or the
application executing a controlled sequence of heap allocations and de-allocations to place
specific data at predictable addresses~\cite{jp-advanced, vudo-malloc}. Because the rest of the
application or even Nail's generated code may contain memory corruption bugs, Nail
carefully handles memory allocations as defense-in-depth to make exploiting such vulnerabilities harder.

When parsing input, Nail uses two separate memory arenas. These arenas allocate memory from the
system allocator in large, fixed size blocks. Allocations are handled linearly and all data in the
arena is zeroed and freed at the same time. Nail uses one arena for data used only during parsing,
including dependent fields and temporary streams and is released before the Nail's parser returns.
The other arena is used to allocate the internal data type returned and is freed by the application
once it is done processing an input. 

Furthermore, the internal representation does not include any references to the input stream, which
can therefore be zeroed immediately after the parser succeeds, so an attacker has to write an
exploit that works without referencing data in the raw input.
% \paragraph{Intermediate representation.}

% Most parser generators, such as Bison, do not have to dynamically allocate temporary data on the
% heap, as they evaluate a semantic action on every rule. However, as our goal is to perform as little
% computation as possible before the input has been validated, and we do not want to mix temporary
% objects with the results of our parse, we use an append-only trace to store intermediate parser
% results.

% Hammer solves this problem by storing a full abstract syntax tree. However, this
% abstract syntax tree is at least an order of magnitude larger than the input,
% because it stores a large tree node structure for each input byte and for each
% rule reduced. This allows Hammer semantic actions to get all of the necessary
% information without ever seeing the raw input stream. However, because we also
% automatically generate our second pass, which corresponds to Hammer's semantic
% actions, we can trust it as much as we trust the parser, and thus can expose it
% to the raw input stream.

% Under this premise, the actions need limited information from the
% recognizer to correctly handle the input stream. In particular, the parser's
% control flow branches only at the choice, repetition, and constant combinators.
% Thus, for each of those combinators, we store the minimum amount of information
% required to reconstruct the syntactic structure of the input.

% The trace is an
% array of integers.
% Whenever the parser encounters a choice, it appends two integers to the trace:
% the number of that choice and
% the length of the trace when it began parsing that choice. When backtracking in
% the input, the parser does not backtrack in the trace. This means that offsets
% within the trace can be used for a Packrat hash table to memoize backtrack-heavy
% parsers.
% When encountering a repetition combinator, the parser records the number of
% times the inner parser succeeded, and when encountering a constant parser of
% variable size, it records how much input was consumed by the constant parser. 

% In a second pass, the parser then allocates the internal representation from an
% arena allocator and binds the fields to values from the input, while following
% the trace to determine how many array fields to parse and which choices to pick.

% \paragraph{.}

% During parsing, dependency fields occur before the context in which they are
% used. The parser stores their values and retrieves them afterwards when
% encountering the combinator that uses them. When generating output, the
% dependency field is first filled with a filler value, then later when the first
% combinator that determines this fields value is encountered, the field is
% overwritten. Any further combinators using this dependency will then validate
% that the dependency field is correct.
