\section{Evaluation}
\label{s:eval}

\begin{figure}
\smaller[0.5]
\begin{verbatim}
labels = <many { @length uint8 | 1..64
                 label n_of @length uint8 }
          uint8 = 0>
compressed_labels = {
  $decompressed transform dnscompress ($current)
  labels apply $decompressed labels
}
question = {
  labels compressed_labels
  qtype uint16 | 1..16
  qclass uint16 | [1,255]
}
answer = {
  labels compressed_labels
  rtype uint16 | 1..16
  class uint16 | [1]
  ttl uint32
  @rlength uint16
  rdata n_of @rlength uint8
}
dnspacket = {
  id uint16
  qr uint1
  opcode uint4
  aa uint1
  tc uint1
  rd uint1
  ra uint1
  uint3 = 0
  rcode uint4
  @qc uint16
  @ac uint16
  @ns uint16
  @ar uint16
  questions n_of @qc question
  responses n_of @ac answer
  authority n_of @ns answer
  additional n_of @ar answer
}
\end{verbatim}
\caption{Nail grammar for DNS packets, used by our prototype DNS server.}
\label{fig:dns-full}
\end{figure}


In our preliminary evaluation of Nail, we try to answer two questions:

\begin{itemize}
\item Can Nail grammars support real-world data formats?
\item How much programmer effort is required to build an
      application that uses Nail for data input and output?
\end{itemize}

\paragraph{Data formats.}

To answer the first question, we implemented two Nail grammars:
one for UTF-16 encoded strings, exposing an array of code points
(shown in Figure~\ref{fig:utf16}), and another for a subset of DNS
packets sent to and from an authoritative name server, without label
compression, as per RFC1035 (shown in Figure~\ref{fig:dns-full}).
Furthermore, our GitHub repository contains other Nail
grammars, such as a TAP network stack that processes Ethernet,
ARP, ICMP, IP, and UDP packets, and the grammar for Nail
itself.\footnote{\url{https://github.com/jbangert/nail/tree/master/examples}}
The results suggest that Nail is a good fit for these data formats.



\paragraph{Programmer effort.}

To answer the second question, we implemented a functioning toy DNS
server.  In particular, we cloned the test DNS server from the Hammer
distribution to Nail.  Hammer ships with a toy DNS server written in 683
lines of code, excluding the Hammer framework itself, that responds to
any valid DNS query with a CNAME record to the domain ``spargelze.it''.
Most of this code is taken up with custom validators, semantic actions,
and data structure definitions, with only 52 lines of code defining the
grammar with Hammer's combinators.

Our DNS server consists of 148 lines of C code and 48 lines of Nail
grammar, and supports a custom zone file format with A, NS, MX, and
CNAME records. The same grammar is used, together with 98 lines of C,
to implement a functional toy clone of the \texttt{host} command-line
tool. However, because our grammar does not yet support DNS label
compression, the latter tool will occasionally reject valid real-world
DNS responses. Both clients have functional anti-spoofing measures.

It is hard to compare the programming effort required to implement
our toy DNS server to that of a real world DNS server, since
we have less functionality, in particular for DNS compression
and additional hint records that real-world DNS servers send.
However, the closest in functionality and intent is Dan Bernstein's
djbdns,\footnote{\url{http://cr.yp.to/djbdns.html}} which aims to be
a minimalist, highly secure DNS server. The latest release of djbdns,
including various support tools, is about 10,000 lines of C as measured by
\texttt{sloccount}.  We expect that it is possible to build a feature-par
version with Nail that is an order of magnitude smaller and intend to
do so.


\paragraph{Other issues.}

As Nail is work in progress, many parts of the implementation, syntax and
design are not complete yet and we do not yet have meaningful performance
or security metrics.

