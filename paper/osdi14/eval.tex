\section{Evaluation}
\label{s:eval}
\begin{figure*}

\label{fig:ethernet}
\end{figure*}
\begin{figure}
\smaller[0.5]
\begin{verbatim}
labels = <many { @length uint8 | 1..64
                 label n_of @length uint8 }
          uint8 = 0>
compressed_labels = {
  $decompressed transform dnscompress ($current)
  labels apply $decompressed labels
}
question = {
  labels compressed_labels
  qtype uint16 | 1..16
  qclass uint16 | [1,255]
}
answer = {
  labels compressed_labels
  rtype uint16 | 1..16
  class uint16 | [1]
  ttl uint32
  @rlength uint16
  rdata n_of @rlength uint8
}
dnspacket = {
  id uint16
  qr uint1
  opcode uint4
  aa uint1
  tc uint1
  rd uint1
  ra uint1
  uint3 = 0
  rcode uint4
  @qc uint16
  @ac uint16
  @ns uint16
  @ar uint16
  questions n_of @qc question
  responses n_of @ac answer
  authority n_of @ns answer
  additional n_of @ar answer
}
\end{verbatim}
\caption{Nail grammar for DNS packets, used by our prototype DNS server.}
\label{fig:dns-full}
\end{figure}

\begin{figure}
\begin{verbatim}
!LITTLE-ENDIAN //Cut for conciseness
fileentry(@crc32 uint32,
 @method uint16,
 @compressed_size uint32, 
 @uncompressed_size uint32) = { 
  uint32 = 0x04034b50
  version uint16
  flags file_flags
  @method_local uint16//[...]
  $compressed transform size_u32 
           ($current @compressed_size)
  $uncompressed transform zip_compression 
           ($compressed @method )
  transform crc_32 ($uncompressed @crc32)
  contents apply $uncompressed many uint8
  transform u16_depend (@method_local @method)//[...]
}
dir_entry($file) = {
//[...]  
  @compression_method uint16      
  mtime uint16
  mdate uint16
  @crc32 uint32
  @compressed_size uint32
  @uncompressed_size uint32
  @file_name_len uint16
  @extra_len uint16
  @comment_len uint16//[...]
  @off uint32
  filename n_of @file_name_len uint8
  extra_field n_of @extra_len uint8
  comment n_of @comment_len uint8
  $content transform offset_u32 ($file @off)
  contents apply $content fileentry(@crc32,
     @compression_method,@compressed_size, 
     @uncompressed_size)
}
end_of_directory($file) = {//[...]
 @directory_size uint32 
 @directory_start uint32
 $dirstr1 transform offset_u32 
     ($filestream @directory_start) 
 $directory_stream transform size_u32 
     ($dirstr1 @directory_size)
 @comment_length uint16
 comment n_of @comment_length uint8
 files apply $directory_stream n_of 
     @total_directory_records dir_entry($file)
}
zip_file = { 
 $file, $directory transform 
    zip_end_of_directory ($current)
  contents apply $end_directory
    end_of_directory($file)
}
\end{verbatim}
\caption{Extracted Nail grammar for ZIP files.}
\label{fig:zip-extract}
\end{figure}

In our preliminary evaluation of Nail, we try to answer four questions:

\begin{itemize}
\item Can Nail grammars support real-world data formats?
\item How much programmer effort is required to build an
      application that uses Nail for data input and output?
\item Do applications handling input and output with Nail suffer from less security vulnerabilities?
\item How fast are systems using Nail parsers as opposed to hand-written input handling? 
\end{itemize}

\subsection{Data formats.}
\label{s:eval-formats}
To answer the first question, we implemented grammars for four protocols with particularly
challenging features and show that Nail is a good fit for these data formats. In
\S\ref{s:eval-effort} we will demonstrate that Nail is easier to use than hand-written parsers and
more convenient than state-of-the-art parser generators for formats that they can handle.

In particular, we show extracts from grammars for the following protocols: 
\begin{itemize}
\item A subset of Ethernet network packets, containing ARP, ICMP and UDP/IP packets. See Figure~\ref{fig:ethernet-extract}.
\item DNS packets as described in RFC 1035, featuring label compression. See Figure~\ref{fig:dns-full}.
\item ZIP files, with compression, check-sums and a complicated procedure to identify the header.
  See Figure~\ref{fig:zip-extract}.
\end{itemize}




one for  DNS queries and respones, as per RFC1035 (shown in, one for the ZIP
archive format and a grammar for Ethernet frames containing ARP, ICMP and UDP packets. We also
parse Nail grammars with Nail, demonstrating that our system can also handle text-based protocols
and languages.  

The results suggest that Nail is a good fit for these data formats. 



\subsection{Programmer effort.}


%code size unzip 6.0
%2821 lines
\label{s:eval-effort}
To answer the second question, we implemented small example applications based on the above
grammars and compared code size with comparable applications that process data manually. We will
also compare a toy DNS server implemented in Nail with a similar toy DNS server provided as an
example for the Hammer parser generator. All code size measurements for C were performed by
SlocCount\cite{sloccount}. \XXX[code size is primitive, but we show orders of magnitude in the same language].


%%DNS: 
\paragraph{DNS.}

The first application we built with Nail, is a toy DNS server inspired by the example shipped with
Hammers DNS server consists of 683 lines of C, excluding the Hammer framework itself, that responds to
any valid DNS query with a CNAME record to the domain ``spargelze.it''.
Most of this code is taken up with custom validators, semantic actions,
and data structure definitions, with only 52 lines of code defining the
grammar with Hammer's combinators.

Our DNS grammar, which is partly reproduced in Figure~\ref{fig:dns-full} consists of a 48 line  Nail
grammar and 64 lines of C implementing DNS label compression. \XXX[Make this smaller]
The grammar describes both the structure of DNS packets (36 lines) and of a simple zone-file format
(16 lines) supporting A, NS, MX and CNAME records.
  
Our simple DNS server, which parses a zone file, listens to incoming DNS requests, parses them and
generates appropriate responses with this Nail grammar, is implemented in 149 lines of C. 

 The same grammar is used, together with 98 lines of C,
to implement a functional toy clone of the \texttt{host} command-line
tool. 

It is hard to compare the programming effort required to implement
our toy DNS server to that of a real world DNS server, since
we have less , in particular the additional hint records that real-world DNS servers send and a
complete lack of configuration options.
However, the closest in functionality and intent is Dan Bernstein's
djbdns,\footnote{\url{http://cr.yp.to/djbdns.html}} which aims to be
a minimalist, highly secure DNS server. The latest release of djbdns,
including various support tools, is about 10,000 lines of C as measured by
\texttt{sloccount}. \XXX[Show that we only count the DNS server parts, not the rest. List some
features we don't have]

\paragraph{ZIP.}

Another particularly tricky file format is the ZIP compressed archive format\cite{pkzip}. The ZIP
format has been associated with many vulnerabilities, and the PROTOS Genome project found numerous
security vulnerabilities in most implementations of ZIP and other archive formats. Furthermore,
Android security has been completely by-passed multiple times through parsing differentials in
handling ZIP files\cite{saurik-masterkey}. \XXX[mention talk by Julia Wolf about zip]. 

ZIP files are normally parsed end-to-beginning. At the end of each ZIP file is an \emph{end-of-directory
header}. This header contains a variable length comment, so it has to be located by scanning
backwards from the end of the file until a magic number and a valid length field is found. Many ZIP
implementations disagree on how to find this header in confusing situations, such as when the
comment contains the valid magic bytes\cite{wolf-berlinsides-zip}.
This end-of-directory header contains the offset and size of the \emph{ZIP directory}, which is an
array of one \emph{directory entry header} per file in the archive.
Each entry stores file meta data, such as file name, compressed and uncompressed size and a checksum,
in addition to the offset of a \emph{local file header}. The local file header duplicates most
information from the directory entry header and the compressed file contents follow it immediately.

Duplicating information made sense when ZIP files were
stored on floppy disks with slow seek times and high fault rates and memory constraints made it
impossible to keep the ZIP directory in memory or the archive was split across multiple disks.
However, care must be taken that the meta-data is consistent. For example, vulnerabilities could occur
if the length in the central directory is used to allocate memory and the length in the local
directory is used to extract without checking that they are equal first.

\XXX[condense]

We wrote a grammar for ZIP, which consists of 92 lines of Nail and 78 lines of C implementing two
stream transforms, one for the DEFLATE compression algorithm with the help of the zlib library and
one for finding the end-of-directory header. This grammar avoids the two pitfalls mentioned above:
Nail forces the programmer to specify an explicit procedure for finding the end-of-directory record
and we use dependent fields to enforce that the duplicated information is consistent.


Our grammar is also very efficient to use. Using our grammar, we build a ZIP file extractor in 50
lines of C. Because more recent versions of ZIP have added more features, such as large file support
and encryption, the closest existing tool in functionality is the historic version 5.4 of the Info-Zip unzip
utility\cite{infozip} that is shipped with most Linux distributions. The entire unzip distribution
is about 46.000 lines of code, which is mostly optimised implementation of various compression
algorithms and other configuration and portability code.

However, unzip isolates the equivalent of our Nail tool in the file extract.c, which parses the ZIP
metadata and calls various decompression routines in other files. This file measures over 1,600
lines of C. 

\paragraph{Network stack.}
Write up this stack.
\subsection{Security.} 
Fuzz DNS. Fuzz ZIP. Design argument that security is hard to show. 

\subsection{Performance.}

Compare DNS requests per second of BIND, DJBDNS and NailDNS on a small zonefile.
Compare extraction performance for ZIP. 

