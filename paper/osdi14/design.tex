\section{Design}
\label{s:design}

Nail's goals are to reduce programmer effort required to safely interact with data formats and
prevent vulnerabilities like those described in \S\ref{s:motivation}. In particular, this means:

\begin{itemize}


\item Using the grammar to define both the \emph{external format} and the \emph{internal
    representation}, allowing the same grammar to be re-used in multiple programs, avoiding
  vulnerabilities like the Android Master Key.

\item Both parsing inputs into internal representations, as well as
      generating outputs from internal representations, without
      requiring the programmer to write any semantic actions. This prevents
      vulnerabilities such as the XNU bug, where format recognition and semantics
      are mixed and interact in unexpected ways.

\item Eliminating redundancy in internal representations, such as
      storing both an explicit length field and an implicit length
      of a container data structure, to provide programmers
      an unambiguous view of the data, to avoid bugs such as in the Python ZIP library.

\item Allow programmers to define grammars for complex real-world
      data formats through well-defined extensibility mechanisms. 

\end{itemize}

\subsection{Overview}

\paragraph{Internal model.}

Nail grammars describe both the \emph{external format} and an
\emph{internal representation} of a protocol, as opposed to existing
grammar languages, which describe only the former and leave it up to
the programmer to construct an internal representation, typically in
the form of an abstract syntax tree.
Nail produces the following from a single, descriptive grammar: 

\begin{CompactItemize}
\item \textit{type declarations} for the internal model,
\item the \textit{parser}, a function to parse a sequence of bytes into an
instance of the model, and
\item the \textit{generator}, a function to create a
sequence of bytes from an instance of the model.
\end{CompactItemize}

\noindent

\paragraph{Semantic bijection.}

In order to generate output, Nail maintains
a \emph{semantic bijection} between the external format and the internal model.
That is, when Nail parses an input into an internal representation, and then
generates output from that representation, the two byte streams (input and
output) will have the same meaning (i.e., be interpreted equivalently by Nail).
However, the byte streams might not be identical.
A bijection in the traditional sense often does not make sense for data
formats. Consider a grammar for a text language that tolerates white space, or a binary protocol
that tolerates arbitrarily long padding.
%\footnote{Say, the physical layer of most communication
%  protocols is a possibly infinite sequence of symbols that are syntactically nil followed by a
%  pre-determined synchronization sequence and the actual contents of the transmission.}
 Program semantics should be independent of the number of padding elements in the input, and Nail therefore
does not expose that information to the programmer. We call such discarded fields \emph{constants}.
Similarly, Nail does not preserve the layout of objects referred to by their offsets. If the grammar
consists of a simple protocol without offset fields, constants, and the like, there is a conventional
bijection between  internal models and valid parser inputs.

\paragraph{Hide redundant information.}
Nail's internal model is designed to hide unneeded and redundant information from the application.
Nail introduces \emph{dependent fields}, which contain data that can be computed during generation
and need to be kept as additional state during parsing. Dependent fields are, for example, used to represent
lengths, offsets, and checksums.

\paragraph{Parser extensions.}
Real-world protocols contain complicated ways of encoding data. Fully representing these in an
intentionally limited model such as our parser language is impractical. Therefore, Nail introduces
\emph{transformations}, which allow arbitrary code by the programmer to interact with the parser and
generator. Nail parsers and generators interact with data through an abstract stream, which allows
reading and writing of bits as re-positioning. Transformations allow the programmer to write
functions in a general purpose language that consume streams and define new temporary streams, while
also reading or writing the values of dependent fields.

Initial versions of Nail's design included a special combinator for handling offset fields,
which consumed a dependent field and applied
a parser at the offset specified therein. However, it proved impossible to foresee all the ways in which a protocol could encode an offset;
for example, some protocols such as PDF and ZIP locate structures by scanning for a magic number starting at the
end of the file or a fixed offset. In nested grammars, offsets are also not necessarily computed from
the beginning of a file or packet.
Nail's transformations allow the programmer to write arbitrary functions that can
handle such structures and streams, which are a generic abstraction for input and
output data that allow the decoded data to be integrated with the rest of the
generated Nail parser.


\begin{figure*}
\begin{tabular}{@{}p{5cm}p{6cm}p{5cm}@{}}
\toprule
\bf Nail grammar & \bf External format & \bf Internal data type in C\\
\midrule
\cc{uint4} & 4-bit unsigned integer.& \lstinline$uint8_t$ \\\hline
\cc{int32 | [1,5..255,512]} & Signed 32 bit integer $x \in \{ 1, 512 \} \vee 5\leq x\leq 255$ &
\lstinline$int32_t$ \\\hline

\cc{uint8=0}& 8-bit constant with value 0.& \lstinline$//Empty$\\\hline

\cc{optional int8|16..} & 8-bit integer $\geq 16$ or nothing & \lstinline$int8_t *$\\\hline

\cc{many int8 | ![0]} & A NUL-terminated string. & 
\begin{minipage}{5cm}
\begin{lstlisting}
struct {
  size_t N_count;
  int8_t *elem;
};
\end{lstlisting}
\end{minipage}\\\hline

\begin{minipage}{5cm}
\begin{verbatim}
{ 
 hours uint8
 minutes uint8
}
\end{verbatim}
\end{minipage}
& Structure with two fields. &
\begin{minipage}{5cm}
\begin{lstlisting}
struct {
  uint8_t hours; 
  uint8_t minutes;
};
\end{lstlisting}
\end{minipage}
\\\hline

\begin{minipage}{5cm}
\verb+<int8 = '"'+\\
A parser definition \textit{p}\\
\verb+int8='"'>+
\end{minipage}
\cc{} & Lowercase string enclosed in quotes. & The data type of {\it p} \\\hline


\begin{minipage}{5cm}
\begin{verbatim}
choose {
  A = uint8 | 1..8
  B = uint16 | 256..
}
\end{verbatim}
\end{minipage}
&\begin{minipage}{6cm} Either an 8-bit integer between 1 and 8 or a 16-bit integer larger than $256$.
  \end{minipage}&
\begin{minipage}{5cm}
\begin{lstlisting}
struct{
  enum {A,B} N_type;
  union {
    uint8_t a;
    uint16_t b;
  };
};
\end{lstlisting}
\end{minipage}\\\hline

\begin{minipage}{5cm}
\begin{verbatim}
{ 
 @valuelen uint16
 value n_of @valuelen uint8
 }
\end{verbatim}
\end{minipage}
&
\begin{minipage}{6cm}
A 16-bit field, followed by a length field, followed by that many bytes.
\end{minipage}
&
\begin{minipage}{5cm}
\begin{lstlisting}
struct{ struct{ 
    size_t N_count;
    uint8_t *elem;
} value;}
\end{lstlisting}
\end{minipage}
\\\hline
\begin{minipage}{5cm}
\vspace{0.5em}
\begin{verbatim}
data transform deflate(
    $current @method)
\end{verbatim}
\end{minipage}
&
Applies a programmer-specified function to create a new stream based on the current input stream and
a dependent field. See
Section~\ref{s:transforms}.&\lstinline+//none+ \\\hline
\verb+apply $stream many +\textsl{p}\textit{/*parser*/}&  Parse $p$ from a different stream, then
resume parsing the current stream& Data type of $p$ \\
\bottomrule
\end{tabular}
\caption{Syntax of Nail parser declarations and the formats and data types they describe.}
\label{fig:syntax}
\end{figure*}

\subsection{Basics}

A Nail parser defines both the structure of some external format and a data type to represent that
format. Parsers are constructed by combinators over simpler parsers, an approach popularized by the Parsec
framework~\cite{LeijenMeijer:parsec}. We provide the most common combinators familiar from other parser
 combinator libraries, such as Parsec and Hammer~\cite{hammer-parser} and extend them so they also
 describe a data type. 

We present both a systematic overview of Nail's syntax with short examples
in Figure~\ref{fig:syntax}, and explain our design in more detail below,
using a grammar for the well-known DNS protocol as a running example
(shown in Figure~\ref{fig:dns-full}).

\begin{figure}
\smaller[0.5]
\begin{verbatim}
labels = <many { @length uint8 | 1..64
                 label n_of @length uint8 }
          uint8 = 0>
compressed_labels = {
  $decompressed transform dnscompress ($current)
  labels apply $decompressed labels
}
question = {
  labels compressed_labels
  qtype uint16 | 1..16
  qclass uint16 | [1,255]
}
answer = {
  labels compressed_labels
  rtype uint16 | 1..16
  class uint16 | [1]
  ttl uint32
  @rlength uint16
  rdata n_of @rlength uint8
}
dnspacket = {
  id uint16
  qr uint1
  opcode uint4
  aa uint1
  tc uint1
  rd uint1
  ra uint1
  uint3 = 0
  rcode uint4
  @qc uint16
  @ac uint16
  @ns uint16
  @ar uint16
  questions n_of @qc question
  responses n_of @ac answer
  authority n_of @ns answer
  additional n_of @ar answer
}
\end{verbatim}
\caption{Nail grammar for DNS packets, used by our prototype DNS server.}
\label{fig:dns-full}
\end{figure}

\paragraph{Rules.}
A Nail grammar consists of rules that assign a parser to a name. Rules are written as assignments,
such as \cc{ints = /*parser definition*/}, which defines a rule called \cc{ints}. As we will
describe later in \S\ref{s:dependent} and \S\ref{s:transforms}, rules can optionally consume parameters.
Rules can be invoked in a Nail grammar anywhere a parser can appear. Rule invocations act as though
the body of the rule had been substituted in the code. If parameters appear, they are passed by
reference.
\paragraph{Integers and constraints.}

Nail's fundamental parsers represent signed or unsigned
integers with arbitrary lengths up to 64 bits.
Note that is possible to define parsers for sub-byte lengths, for example, the flag bits in the DNS
message header, in lines 4 through 8.

The grammar can also constrain the values of an integer.  Nail expresses
constraints as a set of permissible values or value ranges. Extending the Nail language and
implementation to support richer constraints languages would be relatively trivial, however we have
found that the current syntax covers permissible values within existing protocols correctly and
concisely.


\paragraph{Repetition.}

The \emph{many} combinator takes a parser and applies it repeatedly
until it fails, returning an array of the inner parser's results. In line 39 of the DNS grammar, a
sequence of labels is parsed by parsing as many labels as possible, that is, until an invalid length
field is encountered.
The \emph{sepBy} combinator
additionally takes a constant parser, which it applies in between parsing
two values, but not before parsing the first value or after parsing the
last.  This is useful for parsing an array of items delimited by a separator.
%For examle, \cc{many uint8} represents an array of 8-bit unsigned
%integers, and \cc{sepBy uint8=',' (many uint8 | '0'..'9')} recognizes
%comma-separated lists of decimal numbers.

\paragraph{Structures.}

Nail provides a structure combinator with semantic labels instead of the sequence combinator that
other parser combinator libraries use to capture structures in data formats. 
The structure combinator consists of a sequence of fields, typically consisting of a label and a
parser that describes the contents of that field, surrounded by curly braces. Other field types will be described below.
 The syntax of the structure combinator is inspired by the Go language~\cite{golang}, with field
 names preceding their definition.


% A sample grammar for a DNS header and the corresponding C data type the Nail parser produces, are
% shown in Figure~\ref{fig:dns-struct}.

\paragraph{Constants.}
In some cases, not all bytes in a structure actually contain information, such as magic numbers or
reserved fields. Those fields can be represented in Nail grammars by constant fields in structures.
Constant fields do not correspond to a field in the internal model, but they are validated during
parsing and generated during output.  Constants can either have integer values, such as
in line 9 of the DNS grammar, or string values for text-based protocols, e.g. \cc{many uint8 = ``Foo''}.


In some protocols, there might be many ways to represent the same constant field
and there is no semantic difference between the different syntactic representations.
Nail therefore allows repeated constants, such as \cc{many (uint8=' ')}, which parses any number of space characters, or
\cc{|| uint8 = 0x90 || uint16 = 0x1F0F}, which parses two of the many representations for x86
NOP instructions, which are used as padding between basic blocks in an executable. 

As discussed above, choosing to use these combinators on constant parsers
weakens the bijection between the format and the data type, as there are
multiple byte-strings that correspond to the same internal representation and
the generator chooses one of these.

\paragraph{Wrap combinator.} 
When implementing real protocols with Nail, we often found
structures that consist of many constants and only one named field. This pattern is
common in binary protocols which use fixed headers to denote the type of data
structure to be parsed.  In order to keep the internal representation cleaner,
we introduced the wrap combinator, which takes a sequence of parsers containing
exactly one non-constant parser. The external format is defined as though the wrap combinator were a
structure, but the data model does not introduce a structure with just one element, making the application-visible representation
(and thus application code) more concise.
Line 39 of the DNS grammar uses the wrap combinator to hide the terminating NUL-byte of a sequence
of labels.


\paragraph{Choices.}
If multiple structures can appear at a given position in a format, the programmer lists the options
along with a label for each in the \cc{choose} combinator. 
During parsing, Nail remembers the current input position and attempts each option in the order they
appear in the grammar. If an option fails, the parser backtracks to the initial position. If no
options succeed, the entire combinator fails. In the data model, choices are represented as tagged
unions.  The programmer has to be careful when options overlap, because if the programmer meant to
generate output for a choice, but the external representation is also valid for an earlier,
higher-priority option, the parser will interpret it as such. However, real data formats normally do
not have this overlap and we did not encounter it in the grammars we wrote.
An example is provided in Figure~\ref{fig:grammar-arith}.

% If two options the same string could be parsed by two options, generated output for the latter
% option is not necessarily understood identically by the parser. However, actual data formats are
% usually not ambiguous in this sense.
% \XXX[nz][I don't understand this paragraph (esp the first sentence).]


\paragraph{Optional.}
Nail includes an \cc{optional} combinator, which attempts to recognize a value, but succeeds
without consuming input when it cannot recognize that value. Syntactically, \cc{optional} is
equivalent to a choice between the parser and an empty structure, but in the internal model it is
more concisely represented as a reference that is null when the parser fails.
For example, the grammar for  Ethernet headers uses \cc{optional vlan\_header} to parse the VLAN
header that only appears in Ethernet packets transmitted to a non-default VLAN\@.

\paragraph{References.}
Rules allow for recursive grammars. To support recursive data types, we allow introduce the
reference combinator \cc{*}  that does not change the syntax of the external format described, but
introduces a layer of indirection, such as a reference or pointer, to the model data type.
The reference combinator does not need to used when another combinator, such as \cc{optional} or
\cc{many}, already introduces indirection in the data type. An example is shown in Figure~\ref{fig:grammar-arith}.

\begin{figure}
\begin{verbatim}
expr = choose {
  PAREN = <uint8='('; *expr; uint8=')'>
  PRODUCT = sepBy1 uint8='*' expr
  SUM = sepBy1 uint8='+' expr
  INTEGER = many1 uint8 | '0' .. '9'
}
\end{verbatim}
% \cc{paren = < uint8="("; optional *paren; uint8=")" >}.
\caption{Grammar for sums and products of integers.}
\label{fig:grammar-arith}
\end{figure}
% To support recursive structures, we include a reference combinator \cc{*}, which
%   is syntactically equivalent to directly including the parser, but in the internal model adds a
%   level of indirection to the value. This allows recursive structures to be parsed, f



\subsection{Dependent fields}
\label{s:dependent}
Data formats often contain values that are determined by other values or the layout of information,
such as checksums, duplicated information, or offset and  length fields.
We represent such values using \emph{dependent fields} and handle them transparently during
parsing and generation without exposing them to the internal model. 


Dependent fields are defined within a structure like normal fields, but their name starts with an \cc{@} symbol.
A dependent field is in scope and can be referred to by the definition of all subsequent fields in
the same structure. Dependent fields can be passed to rule invocations as parameters.

Dependent fields are handled like other fields when parsing input, but their values are not stored
in the internal data type. Instead the value can be referenced by subsequent parsers and it
discarded when the field goes out of scope.
When generating output, Nail visits a dependent field twice. First, while generating the other fields of a
structure, the generator reserves space for the dependent field in the output. Once
the dependent field goes out of scope, the generator  writes the
dependent field's value to this space.

Nail provides only one built-in combinator that uses dependent fields, \cc{n_of}, which acts like
the \cc{many} combinator, except it represents an exact number, specified in the dependent field, of
repetitions, as opposed as many repetitions as possible. For example, DNS labels, which
are encoded as a length followed by
a value are described in line 38 of the DNS grammar.
Other dependencies, such as offset fields or checksums, are not handled directly by combinators, but
through  transformations, as we describe next.

\subsection{Input streams and transformations}
\label{s:transforms}

Traditional parsers handle input one symbol at a time, from beginning to end.
However, real-world formats often require non-linear parsing. Offset fields require a parser to move
to a different position in the input, possibly backwards. Size fields require the parser to stop
processing before the end of input has been reached, and perhaps resume executing a parent parser.
Other cases, such as compressed data, require more complicated processing on parts of the input
before it can be handled.

Nail introduces two concepts to handle these challenges, \emph{streams} and \emph{transformations}. 
Streams represent a sequence of bytes that contain some external format. The parsers and generators
that Nail generates always operate on an implicit stream named \cc{\$current} that they process front to
back, reading input or appending output.
Grammars can use the \cc{apply} combinator to parse or generate external data on a different stream,
inserting the result in the data model.

Streams are passed as arguments to a rule or defined within the grammar through \emph{transformations}.
The current stream is always passed as an implicit parameter.

Transformations are two arbitrary functions called during parsing and output generation.
The parsing function  takes any number of stream arguments and dependent field values,
and produces any number of temporary streams. This function may reposition and read from the
input streams and read the values of dependent fields, but not change their contents and values. 
The generating function has to be an inverse of the parsing function. It takes the same number of
temporary streams that the parsing function produces, and writes the same number of streams and
dependent field values that the parsing function consumes.

Typically, the top level of most grammars is a rule that takes only a single stream, which may then
be broken up by various transformations and passed to sub-rules, which eventually parse various linear
fragment streams. Upon parsing, these fragment streams are generated and then combined by the
transforms.

 To reduce both programmer effort and the risk of unsafe operations, Nails provides implementations of
transformations for many common features, such as checksums, size, and offset fields. Furthermore,
Nail provides library functions that can be used to safely operate on streams, such as splitting and
concatenation. Nail implements streams as iterators, so they can share underlying buffers and can be
efficiently duplicated and split.

Transformations need to be carefully written, because they can escape Nail's security and introduce
bugs. However, as we will show in \S\ref{s:eval-effort}, Nail transformations are much shorter than
hand-written parsers and many formats can be represented with just the transformations in the
standard library. 
For example, our Zip transformations are 78 lines of code, compared to 1600 lines of code for a
   hand-written parser. Additionally, Nail provides convenient and safe interfaces for allocating
   memory and accessing streams that address the most common occurences of buffer overflow
   vulnerabilities.  

Transformations can handle a wide variety of patterns in data formats, including the following: 
\paragraph{Offsets.}
A built-in transformation for handling offset fields, which is invoked as follows:
\cc{\$fragment transform offset\_u32(\$current, \cc{@}offset)}. 
This transformation corresponds to two functions for parsing and
generation, as shown in Figure~\ref{fig:xform-sig}. It defines a new stream \cc{\$fragment} that can
be used to parse data at the offset contained in \cc{@offset}, by
using \cc{apply \$fragment some_parser}.

\begin{figure}[h]
\smaller[0.5]
\begin{Verbatim}[commandchars=\\\{\},codes={\catcode`\$=3\catcode`\^=7\catcode`\_=8}]
\PY{k+kt}{int} \PY{n+nf}{offset\PYZus{}u32\PYZus{}parse}\PY{p}{(}\PY{n}{NailArena} \PY{o}{*}\PY{n}{tmp}\PY{p}{,}
                 \PY{n}{NailStream} \PY{o}{*}\PY{n}{out\PYZus{}str}\PY{p}{,}
                 \PY{n}{NailStream} \PY{o}{*}\PY{n}{in\PYZus{}current}\PY{p}{,}
                 \PY{k}{const} \PY{k+kt}{uint32\PYZus{}t} \PY{o}{*}\PY{n}{off}\PY{p}{)}\PY{p}{\PYZob{}}
  \PY{c+cm}{/* out\PYZus{}str = suffix of in\PYZus{}current}
\PY{c+cm}{               at offset *off */}
\PY{p}{\PYZcb{}}

\PY{k+kt}{int} \PY{n+nf}{offset\PYZus{}u32\PYZus{}generate}\PY{p}{(}\PY{n}{NailArena} \PY{o}{*}\PY{n}{tmp}\PY{p}{,}
                    \PY{n}{NailStream} \PY{o}{*}\PY{n}{in\PYZus{}fragment}\PY{p}{,}
                    \PY{n}{NailStream} \PY{o}{*}\PY{n}{out\PYZus{}current}\PY{p}{,}
                    \PY{k+kt}{uint32\PYZus{}t} \PY{o}{*}\PY{n}{off}\PY{p}{)}\PY{p}{\PYZob{}}
  \PY{c+cm}{/* *off = position of out\PYZus{}current */}
  \PY{c+cm}{/* append in\PYZus{}fragment to out\PYZus{}current */}
\PY{p}{\PYZcb{}}
\end{Verbatim}

\caption{Pseudocode for two functions that implement the offset transform.}
\label{fig:xform-sig}
\end{figure}


\paragraph{Sizes.}
A similar transformation handles size fields. Just like the offset transform, it takes two parameters, a
stream and a dependent field, but instead of returning the suffix of the current stream after an
offset, it returns a slice of the given size from the current stream starting at its current
position. When generating, it appends the fragment stream to the current stream and writes the size
of the fragment to the dependent field.

\paragraph{Compressed data.}
Encoded, compressed, or encrypted data can be handled transparently by writing a custom
transformation that transforms a coded stream into one that can be parsed by a Nail grammar and vice
versa. This transformation must be carefully written to not have bugs.

\paragraph{Checksums.}
Checksums can be verified and computed in a transformation that takes a stream and a dependent
field. In some cases, a checksum is calculated over a buffer that contains the checksum itself, with
the checksum being set to some particular value. Because the functions implementing a transformation
are passed a pointer to any dependent fields, the checksum function can set the checksum's initial
value before calculating the checksum over the entire buffer, including the checksum.


\paragraph{}
A real-world example with many different transforms, used to support
the ZIP file format, is described in \S\ref{s:eval-formats}.


% For example, we imagine the following grammar could be used to represent
% a sequence of bytes \texttt{data} followed by its CRC32 checksum:

% \begin{verbatim}
% data many uint8; @checksum uint32
% raw_depend @checksum data crc32
% \end{verbatim}

% \noindent
% where \texttt{crc32} is a function supplied by the application, with
% the following signature:

% \begin{verbatim}
% bool crc32(uint32_t *out, uint8_t *in);
% \end{verbatim}

% Because this feature compromises Nail's security guarantees, it should
% only be used in limited circumstances and with carefully prepared checksum
% functions.  This feature is not implemented in the current prototype.

