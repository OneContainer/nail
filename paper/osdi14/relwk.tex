\section{Related work}
\label{s:relwk}


\XXX[Parsifal at langsec '14]
\paragraph{Parsers}

Generating parsers and generators from an executable specification
is the core concept of Interface Generators, e.g. in CORBA or more
recently~\cite{varda2008}. However, while interface generators work
very well for existing grammars, they do not allow full control over the
format of the output, so cannot be used to implement legacy protocols.
Very related work has been done at Bell Labs with the PacketTypes
system~\cite{mccann2000packet}, however PacketTypes works only as a
parser, not as an output generator and does not support the expressive
power of parsing expression grammars (PEGs), but rather implements a
C-like structure model enhanced with  length fields and
constraints. PacketTypes cannot handle complicated encodings such as  compressed data.

Parser generators for binary protocols were first introduced by the
Hammer~\cite{hammer-parser} parser.
While previous parser generators could also be used to write grammars
for binary protocols,\footnote{Theoretically speaking, the alphabet
over which a grammar is an  abstract set, so most algorithms work
just as well on an alphabet of $\{0,1\}$.} doing so is practically
inconvenient. Hammer allows the programmer to specify a grammar in
terms of bits and bytes instead of characters. Common concerns, such as
endianness and bit-packing are handled transparently. 

Hammer implements grammars as language-integrated parser combinators, an approach popularized by
Parsec for Haskell~\cite{LeijenMeijer:parsec}. The parser combinator style (to our knowledge, first
described in~\cite{burge1975recursive}) is a natural way of concisely expressing top-down
grammars~\cite{Danielsson:2010:TPC:1863543.1863585}\footnote{For more background on the history of
  expressing grammars, see Bryan Ford's masters thesis~\cite{ford2002packrat}, which also describes
  the default parsing algorithm used by Hammer.} by composing them from one or multiple sub-parsers.
Hammer then constructs a tree of function pointers which can be invoked to parse a given input into
an abstract syntax tree (AST).

Nail improves upon Hammer in three ways. First, Nail  generates output besides recognizing input.
Second, Nail does not require the programmer to write potentially insecure semantic actions. Last,
Nail's structural dependencies and transformations allow it to work with protocols Hammer cannot recognize, such as
protocols with offset fields or length fields, although Hammer has special facilities for  arrays immediately preceded by their length.

Parsifal~\cite{ANSSI:parsifal} is a parser framework that also supports generating output for Ocaml.
Parsifal structures grammars as an Ocaml type that holds an internal model and functions for parsing
input and output.
However, their generator can only produce parsers and generators for simple, fixed-size structures.
The programmer can then use these when implementing parsers and generators for more
complicated formats, manually handling offsets, checksums and the like, risking bugs. Nail can
handle more complicated constructs without the programmer writing code to support them.

\paragraph{Application use of parsers.}
Generated parsers have long been used to parse human input, such as programming languages and
configuration files. Frequently, such languages are often specified with a formal grammar in an
executable form. Unfortunately,  parser frameworks are seldom used  to recognize
machine-created input.

A notable exception is the Mongrel
web server~\cite{mongrel} which uses a grammar
for HTTP written in the Ragel~\cite{ragel-paper} regular expression
language. Mongrel was re-written from scratch multiple times to achieve
better scalability and design, yet the grammar could be  re-used across
all iterations~\cite{patterson-citation}.

\paragraph{Other vulnerability mitigations.}

Input processing vulnerabilities fall into two broad classes.  The first
class is memory safety bugs, such as buffer overflows, which allow an
adversary to corrupt the application's memory using specially crafted
inputs.  These mistakes arise in lower-level languages that do not
provide memory safety guarantees such as C, and can be partially
mitigated by a wide range of techniques, for example static analysis,
dynamic instrumentation, and address space layout randomization, that
make it more difficult for an adversary to exploit these bugs.
Nail helps developers of lower-level languages avoid these bugs
in the first place.

The second class is logic errors, where application code misinterprets input data. Safe languages
and exploit mitigation technologies do not help against such vulnerabilities. This can lead to
serious security consequences when two systems disagree on the meaning of a network packet or a
signed message, as in iOS\footnote{The XNU kernel and the user-mode code-signing verifier interpret
  executable metadata differently, so the code signature sees different bytes at a virtual address
  than the executable that runs.}~\cite{evaders6} and Android~\cite{saurik-masterkey}code signing
and even the X.509 protocol underlying SSL~\cite{DBLP:conf/fc/KaminskyPS10}. Logic errors are often
causing memory corruption bugs as well, for example when a protocol stores the size of a buffer in
two places and the implementation does not check their consistency. These mistakes are
highly application-specific, and are difficult to mitigate using existing techniques, and these
mistakes can occur even in high-level languages that guarantee memory safety. By allowing developers
to specify their data format just once, Nail avoids logic errors and inconsistencies in parsing and
output generation.

% A subclass of logic errors are so-called \textit{weird machines}, where
% implementation side effects or under-specified parser behavior leads to a
% protocol or data format inadvertently becoming a Turing-complete execution
% environment, even though the original grammar did not require it.
% Frequently, this execution environment can either
% then directly manipulate data in unwanted ways or be used to make
% exploiting another bug feasible.\footnote{For example, by compiling a
% return-oriented-programming exploit from code fragments discoverepd on
% the fly.} Examples include x86 page tables~\cite{bangert2013page}, and
% ELF symbols and relocations~\cite{shapiro2013weird}. In the offensive
% research community, this has been generalized into treating a program
% as a \textit{weird machine}~\cite{bratus2011exploit} that operates on
% an input, analogous to a virtual machine operating on bytecode.
% Nail avoids these problems by having the parser precisely match the
% specified grammar, eliminating under-specified behavior.

% With currently prevalent software design methods, it is hard to isolate
% input-handling code from the rest of the program, as user input is
% typically passed through the program and processed in bits and pieces as a
% ``shotgun parser''~\cite{shotgun-parser}. Thus, partially invalid input might be partly 
% processed by the program until the invalid sections are encountered.


Proper input recognition has been shown to be an excellent way of eliminating ambiguous inputs. In
one case, a PDF parser implemented in Coq rejects over $95\%$ of known malicious
PDFs~\cite{Bogk-PDF}. The same project also produced a ``counter-example'' PDF that triggered
numerous bugs in other PDF parsers. However, manually writing parser code and proving unambiguity
does not scale to the number of file formats and protocols in existence and might result in parser
code tied to one specific application.

\paragraph{Workshop paper}
We presented an early design of Nail at the LangSec workshop~\cite{bangert:nail-langsec} that had
no support for transformations, as described in Section~\ref{s:transforms}, and much more limited
dependent fields, as described in Section~\ref{s:dependent}. With our extended and refined design,
Nail can support much more complicated, real-world data formats.
%%This is supposed to go into design
% While it is possible to express short transformations on the input
% entirely as semantic actions\footnote{This is in fact the design
% rationale, to perform computation on the fly as the parser walks the
% parse tree.}, more complicated programs usually construct an internal
% representation, which contains all relevant information from the input
% in a format native to the programming language used. For example, a C
% programmer ideally wants to deal with structs and NULL-terminated arrays,
% whereas a C++ programmer might expect STL containers, a Java programmer
% interfaces, a Haskell programmer records and a LISP programmer property
% lists.  The structure of this internal representation usually resembles
% the structure of the grammar.

