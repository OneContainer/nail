\begin{figure}
\smaller[1.0]
\begin{Verbatim}[numbers=left]
dnspacket = {
  id uint16
  qr uint1
  opcode uint4
  aa uint1
  tc uint1
  rd uint1
  ra uint1
  uint3 = 0
  rcode uint4
  @qc uint16
  @ac uint16
  @ns uint16
  @ar uint16
  questions n_of @qc question
  responses n_of @ac answer
  authority n_of @ns answer
  additional n_of @ar answer
}
question = {
  labels compressed_labels
  qtype uint16 | 1..16
  qclass uint16 | [1,255]
}
answer = {
  labels compressed_labels
  rtype uint16 | 1..16
  class uint16 | [1]
  ttl uint32
  @rlength uint16
  rdata n_of @rlength uint8
}
compressed_labels = {
  $decompressed transform dnscompress ($current)
  labels apply $decompressed labels
}
label = { @length uint8 | 1..64
           label n_of @length uint8 }
labels = <many label; uint8 = 0>
\end{Verbatim}
\caption{Nail grammar for DNS packets, used by our prototype DNS server.}
\label{fig:dns-full}
\end{figure}

\begin{figure}
\smaller[1.0]
\begin{Verbatim}[commandchars=\\\{\},codes={\catcode`\$=3\catcode`\^=7\catcode`\_=8}]
\PY{k}{struct} \PY{n}{dnspacket} \PY{p}{\PYZob{}}
  \PY{k+kt}{uint16\PYZus{}t} \PY{n}{id}\PY{p}{;}
  \PY{k+kt}{uint8\PYZus{}t} \PY{n}{qr}\PY{p}{;}
  \PY{c+cm}{/* ... */}
  \PY{k}{struct} \PY{p}{\PYZob{}}
    \PY{k}{struct} \PY{n}{question} \PY{o}{*}\PY{n}{elem}\PY{p}{;}
    \PY{k+kt}{size\PYZus{}t} \PY{n}{count}\PY{p}{;}
  \PY{p}{\PYZcb{}} \PY{n}{questions}\PY{p}{;}
\PY{p}{\PYZcb{}}\PY{p}{;}
\end{Verbatim}

\caption{Portions of the C data structures defined by Nail for the DNS
  grammar shown in Figure~\ref{fig:dns-full}.}
\label{fig:dns-full-struct}
\end{figure}

\begin{figure}
\smaller[1.0]
\begin{Verbatim}[commandchars=\\\{\},codes={\catcode`\$=3\catcode`\^=7\catcode`\_=8}]
\PY{k}{struct} \PY{n}{dnspacket} \PY{p}{\PYZob{}}
    \PY{k+kt}{uint16\PYZus{}t} \PY{n}{id}\PY{p}{;}
    \PY{k+kt}{uint8\PYZus{}t} \PY{n}{qr}\PY{p}{;}
    \PY{c+cm}{/*  abbreviated */}
    \PY{k}{struct} \PY{p}{\PYZob{}}
        \PY{n}{question}\PY{o}{*}\PY{n}{elem}\PY{p}{;}
        \PY{k+kt}{size\PYZus{}t} \PY{n}{count}\PY{p}{;}
    \PY{p}{\PYZcb{}} \PY{n}{questions}\PY{p}{;}
\PY{p}{\PYZcb{}}\PY{p}{;} 
\PY{c+cm}{/* Other structures */}
\PY{k}{struct} \PY{n}{dnspacket}\PY{o}{*}\PY{n+nf}{parse\PYZus{}dnspacket}\PY{p}{(}
  \PY{n}{NailArena} \PY{o}{*}\PY{n}{arena}\PY{p}{,} 
  \PY{k}{const} \PY{k+kt}{uint8\PYZus{}t} \PY{o}{*}\PY{n}{data}\PY{p}{,} 
  \PY{k+kt}{size\PYZus{}t} \PY{n}{size}\PY{p}{)}\PY{p}{;}
\PY{k+kt}{int} \PY{n+nf}{gen\PYZus{}dnspacket}\PY{p}{(}\PY{n}{NailArena} \PY{o}{*}\PY{n}{tmp\PYZus{}arena}\PY{p}{,}
  \PY{n}{NailStream} \PY{o}{*}\PY{n}{out}\PY{p}{,}
  \PY{k}{struct} \PY{n}{dnspacket} \PY{o}{*} \PY{n}{val}\PY{p}{)} \PY{p}{;}
\end{Verbatim}

\caption{The API functions generated by Nail for parsing inputs and
  generating outputs for the DNS grammar shown in Figure~\ref{fig:dns-full}.}
\label{fig:dns-full-api}
\end{figure}

