\chapter{Related work}
\label{s:relwk}

This thesis is based on previously published papers about Nail, including \cite{bangert:nail-langsec}, \cite{bangert:nail-osdi14} and \cite{bangert:nail-login}.
We presented an earlier design of Nail at a
workshop~\cite{bangert:nail-langsec}.  At that stage, Nail had only
limited support for dependent fields, and did not support stream
transforms at all, which are crucial for supporting real-world formats
like DNS and ZIP\@.  The workshop paper also did not provide a detailed
design discussion or evaluation.
Since presenting the conference paper~\cite{bangert:nail-osdi14}, we implemented the C++ backend.
\paragraph{Parsers.}

Generating parsers and generators from an executable specification is the
core concept of interface generators, such as CORBA~\cite{omg:corba},
XDR~\cite{RFC:1832}, and Protocol Buffers~\cite{varda2008}.  However,
interface generators do not allow the programmer to specify the byte-level
data format; instead, they define their own data encoding that is
specific to a particular interface generator.  For instance, XDR-based
protocols are incompatible with Protocol Buffers.  Moreover, this means
that interface generators cannot be used to interact with existing protocols
that were not defined using that interface generator in the first place.
As a result, interface generators cannot be used to parse or generate
widely used formats such as DNS or ZIP, which is a goal for Nail.

Closely related work has been done in the field of data description languages, for example
PacketTypes~\cite{mccann2000packet} and DataScript~\cite{back2002datascript}; a
broader overview of data description languages can be found in Fisher et
al~\cite{Fisher:700DDL}.
PacketTypes implements a C-like structure model enhanced with
length fields and constraints, but works only as a parser, and
not as an output generator.
DataScript adds output generation and built-in support for offset fields.
A particularly sophisticated data description language, PADS~\cite{Fisher:2005:PDL:1064978.1065046},
which is targeted more towards offline analysis, even features built-in support for XML and
automatic grammar inference.
However, these systems cannot easily handle complicated encodings such as compressed data, which are
supported by Nail's stream transforms. While sophisticated languages like PADS allow for handling
particular variations of offset fields, compressed data, or even XML entities, each of these
features has to be implemented in the data description language and all associated tools. Nail's
transformations keep the core language small, while enabling the wide range of
features real-world protocols require. 

Recently, the Hammer project~\cite{hammer-parser} introduced a security-focused parser framework for
binary protocols.
Hammer implements grammars as language-integrated parser combinators, an approach popularized by
Parsec for Haskell~\cite{LeijenMeijer:parsec}. The parser combinator style (to our knowledge, first
described by Burge~\cite{burge1975recursive}) is a natural way of concisely expressing top-down
grammars~\cite{Danielsson:2010:TPC:1863543.1863585}
by composing them from one or multiple sub-parsers.\footnote{For more
background on the history of expressing grammars, see Bryan Ford's
masters thesis~\cite{ford2002packrat}, which also describes the
default parsing algorithm used by Hammer.}
Hammer then constructs a tree of function pointers which can be invoked to parse a given input into
an AST\@.

Nail improves upon Hammer in three ways.  First, Nail generates output in
addition to parsing input.  Second, Nail does not require the programmer
to write potentially insecure semantic actions.  Last, Nail's structural
dependencies and stream transforms allow it to work with protocols that
Hammer cannot handle, such as protocols with offset fields, length fields,
checksums, or compressed data, although Hammer has special facilities for
arrays immediately preceded by their length.

Parsifal~\cite{ANSSI:parsifal} is a parser framework for OCaml that also supports
generating output.  Parsifal structures grammars as an OCaml
type that holds an internal model and functions for parsing input and
output.  However, Parsifal can produce parsers and generators
only for simple, fixed-size structures.  The programmer can then use
these when implementing parsers and generators for more complicated
formats, manually handling offsets, checksums, and the like, risking bugs.
Nail handles more complicated constructs without the programmer manually
writing code to support them.



\paragraph{Application use of parsers.}

Generated parsers have long been used to parse human input, such
as programming languages and configuration files. Frequently, such
languages are specified with a formal grammar in an executable
form. Unfortunately, parser frameworks are seldom used to recognize
machine-created input; as we demonstrate in \S\ref{s:eval},
state-of-the-art parser generators are not suitable for parsing or
generating many real-world data formats.

A notable exception is the Mongrel web server~\cite{mongrel} which
uses a grammar for HTTP written in the Ragel regular expression
language~\cite{ragel-paper}.  Mongrel was re-written from scratch
multiple times to achieve better scalability and design, yet the
grammar was reused across all iterations~\cite{patterson-citation}.
We hope that Nail's ideas make it possible to handle a wider range of
protocols using parser generators, and to build more applications on
top of grammar-based parsers.


