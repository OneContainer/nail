\chapter{Evaluation}
\label{s:eval}

\begin{figure*}

\label{fig:ethernet}
\end{figure*}

In our evaluation of Nail, we answer four questions:

\begin{itemize}

\item Can Nail grammars support real-world data formats, and
      are Nail's techniques critical to handling these formats?

\item How much programmer effort is required to build an
      application that uses Nail for data input and output?

\item Does using Nail for handling input and output improve
      application security?

\item Does Nail achieve acceptable performance?

\end{itemize}

\section{Data formats}
\label{s:eval-formats}

To answer the first question, we used Nail to implement grammars
for seven protocols with a range of challenging features.
Figure~\ref{fig:eval-protocols} summarizes these protocols, the lines
of code for their Nail grammars, and the challenging features that make
the protocols difficult to parse with state-of-the-art parser generators.
We find that despite the challenging aspects of these protocols, Nail is
able to capture the protocols, by relying on its novel features: dependent
fields, streams, and transforms.  In contrast, state-of-the-art parser
generators would be unable to fully handle 5 out of the 7 data formats.
In the rest of this chapter, we describe the DNS and Zip grammars in
more detail, focusing on how Nail's features enable us to support these
formats.

\begin{figure}[tb]
\centering
\begin{tabular}{lrl}
\toprule
\textbf{Protocol} & \textbf{LoC} & \textbf{Challenging features} \\ 
\midrule
DNS packets & 48+64 & Label compression,\\
  & & count fields \\
ZIP archives & 92+78 & Checksums, offsets, \\ 
  & & variable length trailer, \\
  & & compression \\
Ethernet  & 16+0\phantom{0} & --- \\
ARP       & 10+0\phantom{0} & --- \\
IP        & 25+0\phantom{0} & Total length field, options \\
UDP       &  7+0\phantom{0} & Checksum, length field \\
ICMP      &  5+0\phantom{0} & Checksum \\
\bottomrule
\end{tabular}

\caption{Protocols, sizes of their Nail grammars, and challenging aspects
of the protocol that cannot be expressed in existing grammar languages.
A + symbol counts lines of Nail grammar code (before the +) and lines of
C code for protocol-specific transforms (after the +).}
\label{fig:eval-protocols}
\end{figure}

\paragraph{DNS.}

In Chapter~\ref{s:design}, we introduced Nail's syntax with a grammar
for DNS packets, shown in Figure~\ref{fig:dns-full}.
The grammar corresponds almost directly to the diagrams in RFC 1035,
which defines DNS~\cite[\S4]{RFC:1035}.  Each DNS packet consists of a
header, a set of question records, and a set of answer records. Domain
names in both queries and answers are encoded as a sequence of labels,
terminated by a zero byte.  Labels are Pascal-style strings, consisting
of a length field followed by that many bytes comprising the label.

One challenging aspect of DNS packets lies in the count fields (\cc{qc},
\cc{ac}, \cc{ns}, and \cc{ar}), which represent the number of questions
or answers in another part of the packet.  Nail's \cc{n_of} combinator
handles this situation easily, which would have been difficult to handle
for other parsers.

Another challenging aspect of DNS is label
compression~\cite[\S4.1.4]{RFC:1035}.  Label compression is used to reduce
the size overhead of including each domain name multiple times in a DNS
reply (once in the question section, and at least once in the response
section).  If a domain name suffix is repeated, instead of repeating that
suffix, the DNS packet may write a two-bit marker sequence followed by
a 14-bit offset into the packet, indicating the position of where that
suffix was previously encoded.


Handling label compression in existing tools, such as Bison or Hammer,
would be awkward at best, because some ad-hoc trick would have to
be used to re-position the parser's input stream.  Keeping track of the
position of all recognized labels would not be enough, as the offset field
may refer to any byte within the packet, not just the beginning of labels.
For this reason, the DNS server used as the example for Hammer does not
support compression.

\begin{figure}
\smaller[0.5]
\begin{Verbatim}[commandchars=\\\{\},codes={\catcode`\$=3\catcode`\^=7\catcode`\_=8}]
\PY{k+kt}{int}
\PY{n+nf}{dnscompress\PYZus{}parse}\PY{p}{(}\PY{n}{NailArena} \PY{o}{*}\PY{n}{tmp}\PY{p}{,}
                  \PY{n}{NailStream} \PY{o}{*}\PY{n}{out\PYZus{}decomp}\PY{p}{,}
                  \PY{n}{NailStream} \PY{o}{*}\PY{n}{in\PYZus{}current}\PY{p}{)}\PY{p}{;}

\PY{k+kt}{int}
\PY{n+nf}{dnscompress\PYZus{}generate}\PY{p}{(}\PY{n}{NailArena} \PY{o}{*}\PY{n}{tmp}\PY{p}{,}
                     \PY{n}{NailStream} \PY{o}{*}\PY{n}{in\PYZus{}decomp}\PY{p}{,}
                     \PY{n}{NailStream} \PY{o}{*}\PY{n}{out\PYZus{}current}\PY{p}{)}\PY{p}{;}
\end{Verbatim}

\caption{Signatures of stream transform functions for handling DNS label compression.}
\label{fig:dns-xform}
\end{figure}

In contrast, Nail is able to handle label compression, by using a stream
transform; the signatures of the two transform functions are shown
in Figure~\ref{fig:dns-xform}.  When parsing a packet, this transform
decompresses the DNS label stream by following the offset pointers.
When generating a packet, this transform receives the current suffix as
an input, and scans the packet so far for previous occurrences, which
implements compression.


\paragraph{ZIP files.}

An especially tricky data format is the ZIP compressed archive
format~\cite{pkzip}.  ZIP files are normally parsed end-to-beginning. At
the end of each ZIP file is an \emph{end-of-directory header}. This header
contains a variable-length comment, so it has to be located by scanning
backwards from the end of the file until a magic number and a valid
length field is found. Many ZIP implementations disagree on how to find
this header in confusing situations, such as when the comment contains
the magic number~\cite{wolf:berlinsides-zip}. 

 This end-of-directory
header contains the offset and size of the \emph{ZIP directory}, which
is an array of \emph{directory entry headers}, one for every file
in the archive.  Each entry stores file metadata, such as file name,
compressed and uncompressed size, and a checksum, in addition to the
offset of a \emph{local file header}. The local file header duplicates
most information from the directory entry header. The compressed file
contents follow the header immediately.

\begin{figure}
\begin{verbatim}
!LITTLE-ENDIAN //Cut for conciseness
fileentry(@crc32 uint32,
 @method uint16,
 @compressed_size uint32, 
 @uncompressed_size uint32) = { 
  uint32 = 0x04034b50
  version uint16
  flags file_flags
  @method_local uint16//[...]
  $compressed transform size_u32 
           ($current @compressed_size)
  $uncompressed transform zip_compression 
           ($compressed @method )
  transform crc_32 ($uncompressed @crc32)
  contents apply $uncompressed many uint8
  transform u16_depend (@method_local @method)//[...]
}
dir_entry($file) = {
//[...]  
  @compression_method uint16      
  mtime uint16
  mdate uint16
  @crc32 uint32
  @compressed_size uint32
  @uncompressed_size uint32
  @file_name_len uint16
  @extra_len uint16
  @comment_len uint16//[...]
  @off uint32
  filename n_of @file_name_len uint8
  extra_field n_of @extra_len uint8
  comment n_of @comment_len uint8
  $content transform offset_u32 ($file @off)
  contents apply $content fileentry(@crc32,
     @compression_method,@compressed_size, 
     @uncompressed_size)
}
end_of_directory($file) = {//[...]
 @directory_size uint32 
 @directory_start uint32
 $dirstr1 transform offset_u32 
     ($filestream @directory_start) 
 $directory_stream transform size_u32 
     ($dirstr1 @directory_size)
 @comment_length uint16
 comment n_of @comment_length uint8
 files apply $directory_stream n_of 
     @total_directory_records dir_entry($file)
}
zip_file = { 
 $file, $directory transform 
    zip_end_of_directory ($current)
  contents apply $end_directory
    end_of_directory($file)
}
\end{verbatim}
\caption{Extracted Nail grammar for ZIP files.}
\label{fig:zip-extract}
\end{figure}


Duplicating information made sense when ZIP files were stored on floppy
disks with slow seek times and high fault rates, and memory constraints
made it impossible to keep the ZIP directory in memory or the archive
was split across multiple disks.  However, care must be taken that the
metadata is consistent. For example, vulnerabilities could occur if
the length in the central directory is used to allocate memory and the
length in the local directory is used to extract without checking that
they are equal first, as was the case in the Python ZIP
library~\cite{python-bug:20078}.
\begin{comment}
\begin{figure}[tb]
\smaller[0.5]
\begin{Verbatim}[commandchars=\\\{\},codes={\catcode`\$=3\catcode`\^=7\catcode`\_=8}]
\PY{k+kt}{int} \PY{n+nf}{zip\PYZus{}end\PYZus{}of\PYZus{}directory\PYZus{}parse}\PY{p}{(}
  \PY{n}{NailArena} \PY{o}{*}\PY{n}{tmp}\PY{p}{,} \PY{n}{NailStream} \PY{o}{*}\PY{n}{out\PYZus{}files}\PY{p}{,}
  \PY{n}{NailStream} \PY{o}{*}\PY{n}{out\PYZus{}dir}\PY{p}{,} \PY{n}{NailStream} \PY{o}{*}\PY{n}{in\PYZus{}current}\PY{p}{)}\PY{p}{;}
\PY{k+kt}{int} \PY{n+nf}{zip\PYZus{}end\PYZus{}of\PYZus{}directory\PYZus{}generate}\PY{p}{(}
  \PY{n}{NailArena} \PY{o}{*}\PY{n}{tmp}\PY{p}{,} \PY{n}{NailStream} \PY{o}{*}\PY{n}{in\PYZus{}files}\PY{p}{,}
  \PY{n}{NailStream} \PY{o}{*}\PY{n}{in\PYZus{}dir}\PY{p}{,} \PY{n}{NailStream} \PY{o}{*}\PY{n}{out\PYZus{}current}\PY{p}{)}\PY{p}{;}
\end{Verbatim}

\caption{Signatures of stream transform functions for handling the
end-to-beginning structure of ZIP files.}
\label{fig:zip-eod-xform}
\end{figure}
\end{comment}
Figure~\ref{fig:zip-extract} shows an abbreviated version of our ZIP
file grammar.  The ZIP grammar is a good example of how transformations
capture complicated syntax in a real-world file format; existing parser
languages cannot handle a file format of this complexity.

The \cc{zip\_file} grammar first splits the entire file stream into two
streams based on the \cc{zip\_end\_of\_directory} transform on line 2.
%(whose two C function signatures are shown in Figure~\ref{fig:zip-eod-xform}). 
The corresponding C function
\cc{zip\_end\_of\_directory_parse}
finds the end-of-directory header as described above, by scanning the
file backwards, and splits the file into two streams, one containing
the end-of-directory header and one containing the file contents. The
\cc{end\_of\_directory} rule is then applied to the header stream in
line 4. All offsets in the ZIP file refer to the beginning of the
file, so the stream \cc{\$file} which contains the file contents
without the header is passed as an argument to all parsers from
hereon. 

The directory header contains the offset and size of the ZIP
directory (lines 9 and 10). The \cc{offset} and \cc{size}
transformations extract a stream containing just the directory from
the file contents. This stream is then parsed as an array of directory
entries in line 17.
Each directory entry in turn points to a local file header, which is
similarly extracted and parsed with the \cc{file} rule.

The \cc{file} rule starting at line 41, describing a ZIP file entry,
takes dependent field
parameters containing file metadata information from the directory
header. However, this same information is duplicated in the file
entry, so the grammar uses the Nail-supplied \cc{u16\_depend}
transform to check whether the two values are equal. Unlike most other
transforms, \cc{u16\_depend} does not consume or produce strings; it
only checks that two dependent fields are equal when parsing, and
assigns the value of the second field to the first when generating.
This ensures that the programmer does not have to worry about
inconsistencies when handling the internal representation of a ZIP
file.

Immediately following the file entry is the compressed data.  Because
most compression algorithms operate on unbounded streams of data, Nail
decompresses data in two steps.  First, it isolates the compressed
data from the rest of the stream by using the \cc{size} transform,
which operates on the current stream, meaning it will consume data
starting at the current position of the parser in the input.  Second,
Nail invokes a custom \cc{zip\_compression} transform that implements
the appropriate compression and decompression functions based on the
specified compression method. These functions are otherwise oblivious
to the layout or metadata of the file.


\section{Programmer effort}
\label{s:eval-effort}

%code size unzip 6.0
%2821 lines

To evaluate how much programmer effort is required to build an
application that uses Nail, we implemented two applications---a
DNS server and an
\cc{unzip} program---based on the above grammars, and compared code
size with comparable applications that process data manually, using
\cc{sloccount}~\cite{sloccount}.  We also compare the code size of our
DNS server to a DNS server written using the Hammer parsing framework,
although it does not fully support DNS (e.g., it lacks label compression,
among other things).  Figure~\ref{fig:effort} summarizes the results.

\begin{figure}[tb]
\centering
\smaller[0.5]
\begin{tabular}{lrr@{~}l}
\toprule
\textbf{Application}
  & \textbf{LoC w/ Nail}
  & \multicolumn{2}{c}{\textbf{LoC w/o Nail}} \\
\midrule
DNS server
  & 295
  & 683
  & (Hammer parser) \\

%  &
%  & 10,000
%  & (DJBDNS) \\

\cc{unzip}
  & 220
  & 1,600
  & (Info-Zip) \\
\bottomrule
\end{tabular}
\caption{Comparison of code size for three applications written in
  Nail,and a comparable existing implementation without Nail.}
\label{fig:effort}
\end{figure}

\paragraph{DNS.}

Our DNS server parses a zone file, listens to incoming DNS requests,
parses them, and generates appropriate responses.  The DNS server is
implemented in 183 lines of C, together with 48 lines of Nail grammar
and 64 lines of C code implementing stream transforms for DNS label
compression.  In comparison, Hammer~\cite{hammer-parser} ships with a toy
DNS server that responds to any valid DNS query with a CNAME record to the
domain ``spargelze.it''.  Their server consists of 683 lines of C, mostly
custom validators, semantic actions, and data structure definitions, with
 52 lines of code defining the grammar with Hammer's combinators.
Their DNS server does not implement label compression, zone files, etc.
From this, we conclude that Nail leads to much more compact code for
dealing with DNS packet formats.

% It is hard to compare the programming effort required to implement
% our DNS server to that of a real world DNS server, since we implement
% less functionality: we do not send the additional glue records and
% we did not implement any configuration options.  However, the closest
% in functionality and intent is Dan Bernstein's djbdns~\cite{djbdns},
% which aims to be a minimalist, highly secure, authoritative-only DNS
% server. The latest release of djbdns, including various support tools,
% is about 10,000 lines of C\@.  \XXX[nz][so what does this mean?]

\paragraph{ZIP.}

We implemented a ZIP file extractor in 50 lines of C code, together with
92 lines of Nail grammar and 78 lines of C code implementing two stream
transforms (one for the DEFLATE compression algorithm with the help of
the zlib library, and one for finding the end-of-directory header).

Because more recent versions of ZIP have added more features, such
as large file support and encryption, the closest existing tool
in functionality is the historic version 5.4 of the Info-Zip unzip
utility~\cite{infozip} that is shipped with most Linux distributions.
The entire unzip distribution is about 46,000 lines of code, which is
mostly optimized implementations of various compression algorithms and
other configuration and portability code.  However, unzip isolates the
equivalent of our Nail tool in the file \cc{extract.c}, which parses the
ZIP metadata and calls various decompression routines in other files. This
file measures over 1,600 lines of C, which suggests that Nail is highly
effective at reducing manual input parsing code, even for the complex
ZIP file format.

\section{Security} 

We use a twofold approach to evaluate the security of applications
implemented with Nail. First, we analyze a list of CVE's related to the
ZIP file format and argue how our ZIP tools based on Nail are immune
against those vulnerability classes. Second, we present the results of
fuzz-testing our DNS server.

\paragraph{ZIP analysis.}

In \S\ref{s:motivation}, we presented 15 input handling vulnerabilities
related to ZIP files.

11 of these vulnerabilities involved memory corruption during input handling.
Because Nail's generated code checks offsets before reading and does not
expose any untrusted pointers to the application, it is immune to
memory corruption attacks by design.

Nail also protects against parsing inconsistency vulnerabilities like the four others we studied.
Nail grammars explicitly encode duplicated information such as the redundant length fields in ZIP
that caused a vulnerability in the Python ZIP library. The other three vulnerabilities exist
because multiple implementations of the same protocol disagree on some inputs. Hand-written protocol
parsers are not very reusable, as they build application-specific data structures and are tightly
coupled to the rest of the code. Nail grammars, however, can be re-used between applications,
avoiding protocol misunderstandings.

\paragraph{DNS fuzzing.}

To provide additional assurance that Nail parsers are free of memory
corruption attacks, we ran the DNS fuzzer provided with the Metasploit
framework~\cite{mspframework} on our DNS server, which sent randomly
corrupted DNS queries to our server for 4 hours, during which it did
not crash or trigger the stack or heap corruption detector.


\section{Performance}

To evaluate whether Nail-based parsers are compatible with good
performance, we compare the performance of our DNS server to that of ISC
BIND 9 release 9.9.5~\cite{bind9}, a mature and widely used DNS server.
We simulate a load resembling that of an authoritative name server.
First, we generate domain names consisting of one or two labels randomly
selected from an English dictionary, and one label that is one of
three popular top-level domains (\cc{com}, \cc{net}, and \cc{org}).
Second, we randomly selected 90\% of these domains and created a zone
file that mapped these domain names to \cc{127.0.0.1}.  Finally, we used
the \cc{queryperf} tool provided with BIND to query each domain between
zero and three times, using a DNS server running on the local machine.
We used a single core of an Intel i7-3610QM system with 12GB of RAM\@.
The benchmark tool kept at most 20 queries outstanding at once, and
was configured to repeat the same randomized sequence of queries for
one minute.  We repeated each test seven times with 50,000 domain names,
restarting each daemon in between; we also repeated the tests with 1
million domain names, and found similar results.  We also performed one
initial dry run to warm the file system cache for the zone file.

The results are shown in Figure~\ref{fig:perf-dns}, and demonstrate
that our Nail-based DNS server can achieve higher performance and
lower latency than BIND\@.  Although BIND is a more sophisticated DNS
server, and implements many features that are not present in our
Nail-based DNS server and that allow it to be used in more complicated
configurations, we believe our results demonstrate that Nail's parsers
are not a barrier to achieving good performance.  Previous papers on
Nail presented results using a Nail parser generated as C, whereas
these results were produced on the same configuration with the newer
C++ parser generator, which improves performance by using a more
efficient stream representation.
\begin{figure}
\begin{tikzpicture}
  \begin{axis}[
      width=0.9\textwidth,
  y=0.75cm,
  ytick={1,2},
  yticklabels={NailDNS, Bind 9},
  xmin=0,
  xlabel ={Throughput},
  x unit ={\frac{\text{queries}}{s}},
  scaled x ticks = false,
  x tick label style = {/pgf/number format/fixed}
]
\addplot+[boxplot]
  table[row sep=\\,y index=0] {
data\\
 140950.177719\\
 137100.337896\\
 140575.256869\\
 139215.019250\\
 109667.926913\\
 125123.464186\\
 139507.888042\\
 };
\addplot+[boxplot]
  table[row sep=\\,y index=0] {
data\\
50713.068173\\
45230.492090\\
47824.395302\\
39167.055247\\
47234.594953\\
59239.056531\\
40436.939492\\
 };
\end{axis}
\end{tikzpicture}

\begin{tikzpicture}
\begin{axis}[
    y=0.75cm,
    width=0.9\textwidth,
  ytick={1,2},
  yticklabels={NailDNS, Bind 9},
  xmin=0,
  xlabel ={Round Trip Time},
  x unit= {msec},
  scaled x ticks = false
]
\addplot+[boxplot]
  table[row sep=\\,y index=0] {
data\\
0.137\\
0.141\\
0.138\\
0.139\\
0.176\\
0.155\\
0.139\\
 };
\addplot+[boxplot]
  table[row sep=\\,y index=0] {
data\\
0.388\\
0.434\\
0.411\\
0.501\\
0.416\\
0.333\\
0.486\\
 };
\end{axis}
\end{tikzpicture}

\caption{A box plot comparing the performance of the Nail-based DNS server compared to
  BIND 9.5.5 on 50,000 domains.  The boxes show the interquartile range, with the middle showing the
  median result. The dots show outliers.}
\label{fig:perf-dns}

\end{figure}


