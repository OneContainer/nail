\chapter{Implementation}
\label{s:impl}

The current prototype of the Nail parser generator can emit parsers in either C or C++. The implementation parses Nail grammars with Nail itself, using a
130-line Nail grammar feeding into a 2,000-line C++ program that
emits the parser and generator code. Bootstrapping is
performed with a subset of the grammar implemented using conventional grammars. 
In this chapter, we will discuss some particular features of our parser
implementation.


A generated Nail parser makes two passes through the input: the first to
validate and recognize the input, and the second to bind this data to the internal
model. Currently the parser uses a straightforward top-down algorithm, which can perform poorly on
grammars that backtrack heavily. However, Packrat
parsing~\cite{packrat-parsing:icfp02} that achieves linear time even in the worst case can be added relatively easily.

\paragraph{Parser input/output.}
Nail-generated parsers interact with their input through an abstract stream data type, which must support the operations show in Figure~\ref{fig:stream-api}. In the C implementation, the programmer can provide an implementation of these operations, or fall back to using a default implementation backed by continuous buffers in memory. Both the input stream and all intermediate streams must use the same implementation.

While the concrete implementation is transparent to the Nail parser itself, transformations can and often must modify the stream data type through other interfaces. The standard library of Nail transformations is at present only for memory-buffer backed streams and often directly manipulates the underlying buffers. This makes many operations, such as splitting a stream into multiple substreams, much faster than if the substreams had to be copied directly. 

\begin{figure*}
  \begin{tabular}{lp{2.75in}}
    \toprule
    \bf Method & \bf Description \\
    \midrule
    \texttt{bool check(uint n)} & Check if the current position is at least $n$ bits before the end of the stream. \\
    \texttt{uint read\_bits(uint n)} &  Read $n$ bits from the stream. Check($n$) must be true. Does not move the current position.\\
    \texttt{uint read\_bits\_little(uint n)} &  Like read\_bits, but interprets the value as little endian.\\
    \texttt{void advance(uint n)} & Advances the current position by $n$ bits.\\
    \texttt{void backup(uint n))} & Reverses the current position by $n$ bits.\\
    \texttt{pos\_t get\_pos(uint n))} & Returns the current position.\\
    \texttt{void reposition(pos\_t old))} & Reverts the current position to a value formerly returned from get\_pos().\\

    \bottomrule
    \end{tabular}
\label{fig:stream-api}
\caption{The operations Nail parsers invoke on their input streams.}
\end{figure*}

\paragraph{C++ implementation.}
The C implementation forces all streams to have the same underlying data type. To address this shortcoming, we developed an optional C++ backend for Nail. Nail C++ parsers are template functions that are parametrized by the types of the streams they operate on. Transformations are template functors -- classes that can perform different computation  and produce a different type of stream depending on the types of their inputs.

Useful transformations are often only possible to write when specialized to their input types, and the fully general will either be very slow or not even implemented. Consider for example the standard library offset transform, which creates a substream starting at a given offset. Given just the stream interface shown in Figure~\ref{fig:stream-api}, the only way to implement offset would be to either to make a copy of the underlying stream, using reposition and read, or to reposition the underlying stream before and after each read. By specializing the transformation for in-memory streams, we can construct another memory stream that points to the correct range in the same memory. Similar specializations could exist for file-descriptor backed streams using a duplicate descriptor or sparse-fragment streams.


Transformations can also be expressed through chaining together other transformations -- which is equivalent to composing them directly in the Nail grammar, but results in shorter and easier to understand Nail grammars.
In addition to allowing complicated file formats to be captured more naturally by mapping their layout onto different stream types, for example, fragmented network packets can be stored as individual fragments and do not have to be assembled into a continuous buffer.

\paragraph{Defense-in-depth.}

Security exploits often rely on raw inputs being present in memory~\cite{shotgun-parser}, for example to include
shell-code or crafted stack frames for ROP~\cite{phrack58:4-nergal} attacks in padding fields or the
application executing a controlled sequence of heap allocations and de-allocations to place
specific data at predictable addresses~\cite{jp-advanced, vudo-malloc}. Because the rest of the
application or even Nail's generated code may contain memory corruption bugs, Nail
carefully handles memory allocations as defense-in-depth to make exploiting such vulnerabilities harder.

When parsing input, Nail uses two separate memory arenas. These arenas allocate memory from the
system allocator in large, fixed-size blocks. Allocations are handled linearly and all data in the
arena is zeroed and freed at the same time. Nail uses one arena for data used only during parsing,
including dependent fields and temporary streams; this arena is released before the parser returns.
The other arena is used to allocate the internal data type returned and is freed by the application
once it is done processing an input. 

Furthermore, the internal representation does not include any references to the input stream, which
can therefore be zeroed immediately after the parser succeeds, so an attacker has to write an
exploit that works without referencing data from the raw input.

Finally, Nail performs sophisticated error handling only in a special debug configuration and will print
error messages about the input only to \cc{stderr}. Besides complicating the parser, advanced error handling
invites programmers to attempt to fix malformed input, such as adding reasonable defaults for a
missing field. Such error-fixing not only introduces parser inconsistencies, but also might allow an
attacker to sneak inconsistent input past a parser.
% \paragraph{Intermediate representation.}

% Most parser generators, such as Bison, do not have to dynamically allocate temporary data on the
% heap, as they evaluate a semantic action on every rule. However, as our goal is to perform as little
% computation as possible before the input has been validated, and we do not want to mix temporary
% objects with the results of our parse, we use an append-only trace to store intermediate parser
% results.

% Hammer solves this problem by storing a full abstract syntax tree. However, this
% abstract syntax tree is at least an order of magnitude larger than the input,
% because it stores a large tree node structure for each input byte and for each
% rule reduced. This allows Hammer semantic actions to get all of the necessary
% information without ever seeing the raw input stream. However, because we also
% automatically generate our second pass, which corresponds to Hammer's semantic
% actions, we can trust it as much as we trust the parser, and thus can expose it
% to the raw input stream.

% Under this premise, the actions need limited information from the
% recognizer to correctly handle the input stream. In particular, the parser's
% control flow branches only at the choice, repetition, and constant combinators.
% Thus, for each of those combinators, we store the minimum amount of information
% required to reconstruct the syntactic structure of the input.

% The trace is an
% array of integers.
% Whenever the parser encounters a choice, it appends two integers to the trace:
% the number of that choice and
% the length of the trace when it began parsing that choice. When backtracking in
% the input, the parser does not backtrack in the trace. This means that offsets
% within the trace can be used for a Packrat hash table to memoize backtrack-heavy
% parsers.
% When encountering a repetition combinator, the parser records the number of
% times the inner parser succeeded, and when encountering a constant parser of
% variable size, it records how much input was consumed by the constant parser. 

% In a second pass, the parser then allocates the internal representation from an
% arena allocator and binds the fields to values from the input, while following
% the trace to determine how many array fields to parse and which choices to pick.

% \paragraph{.}

% During parsing, dependency fields occur before the context in which they are
% used. The parser stores their values and retrieves them afterwards when
% encountering the combinator that uses them. When generating output, the
% dependency field is first filled with a filler value, then later when the first
% combinator that determines this fields value is encountered, the field is
% overwritten. Any further combinators using this dependency will then validate
% that the dependency field is correct.
