%% This file is for producing a Doctoral Thesis proposal.  It should be fairly
%% self-explanatory.  

\documentclass[a4paper]{article}

\usepackage[doublespacing]{setspace}
\usepackage[margin=1in]{geometry}
\usepackage{url}
\begin{document}


\bibliographystyle{plain}
\pagestyle{empty}
\markboth{{\sc thesis proposal}}{{\sc thesis proposal}}
\def\title{Nail, a practical tool for parsing and generating data formats.}
\def\author{Julian Bangert}
\def\addrone{550 Memorial Drive, 15E3}
\def\addrtwo{Cambridge, MA  02139}

\def\degree{Master of Science}
%% Added \deptname for PhD proposals in other fields.
%% Krishna Sethuraman (1990)
\def\deptname{Electrical Engineering \\ and Computer Science}
\def\laboratory{Computer Science and Artificial Intelligence Laboratory}

\def\submissiondate{\today}
\def\completiondate{September 2015}

\def\supervisor{Professor Nickolai Zeldovich}
\def\supertitleone{Associate Professor }



\def\abstract{ The proposed research is extending grammar-based parser
  generators to handle binary protocols and data formats.  Our
  prototype tool, Nail, introduces several new ideas to achieve that
  goal, including dependent fields and stream transforms to capture
  protocol features such as size and offset fields, checksums, and
  compressed data, which are impractical to express in existing
  protocol languages.  We will show several grammars for complicated
  real-world formats and demonstrate applications built with those. We
  argue that applications built with Nail are more resilient to memory
  corruption attacks as well as less-understood parser differential
  vulnerabilites.  }

%%%%%%%%%%%%%%%%%%%%%%%%%%%%%%%%%%%%%%%%%%%%%%%%%%%%%%%%%%%%%%%%%%%%%%%%%%%%
%%%%%%%%%% You Should Not Need To Modify Anything Below Here %%%%%%%%%%%%%%%
%%%%%%%%%%%%%%%%%%%%%%%%%%%%%%%%%%%%%%%%%%%%%%%%%%%%%%%%%%%%%%%%%%%%%%%%%%%%

%%%%%%%%%%%%%%%%%%%%%%%%%%%%%%%%%
%%% Cover Page - Author signs %%%
%%%%%%%%%%%%%%%%%%%%%%%%%%%%%%%%%
\begin{singlespace}
\begin{center}
{\Large \bf 
   Massachusetts Institute of Technology
\\ Department of \deptname \\}
\vspace{.25in}
{\Large \bf
   Proposal for Thesis Research in Partial Fulfillment
\\ of the Requirements for the Degree of
\\ \degree \\}
\end{center}

\vspace{.5in}

\def\sig{{\small \sc (Signature of Author)}}

\begin{tabular}{rlc}
   {\small \sc Title:}                       & \multicolumn{2}{l}{\title}
\\ {\small \sc Submitted by:}
                            & \author  & \\
                            & \addrone & \\
& \addrtwo & \\
& \\
& \\
\cline{2-2} & \\
			    & \makebox[2in][c]{\sig} & \\
\\ {\small \sc Date of Submission:}          & \multicolumn{2}{l}{\submissiondate}
\\ {\small \sc Expected Date of Completion:} & \multicolumn{2}{l}{\completiondate}
\\ {\small \sc Laboratory:}                  & \multicolumn{2}{l}{\laboratory}
\end{tabular}


\vspace{.75in}
{\bf \sc Brief Statement of the Problem:}

\abstract


\underline{\bf  Supervision Agreement}

\vspace{.25in}

\vspace{.25in}
The program outlined in this proposal is adequate for a Master's thesis. The supplies and facilities
required are available, and I am willing to supervise the research and evaluate the thesis report.
\vspace{.5in}
\begin{tabular}{crc}
  \hspace{2in} & {\sc Signed:} & \\ \cline{3-3}
               &               & {\small \sc \supervisor} \\
               &               &                             \\
               & {\sc Date:}   & \\ \cline{3-3}
\end{tabular}

\end{singlespace}
\newpage

\section{Motivation}
Binary file formats and network protocols are hard to parse safely: 
The libpng image decompression library had 24 remotely
exploitable vulnerabilities from 2007 to 2013 according to CVEDetails,
Adobe's PDF and Flash viewers have been notoriously plagued by input
processing vulnerabilities, and even the zlib compression library had
input processing vulnerabilities in the past.
Most of these attacks involve memory corruption; therefore, it is easy to assume that 
solving memory corruption will end all our woes when handling untrusted inputs. 

However, as memory-safe languages and exploit mitigation  tricks are becoming more prevalent,
attackers are moving to a new class of attack called {\em parser differentials}. Many applications use
hand-written input processing code, which is often mixed with the rest of the application, e.g. by
passing a pointer to the raw input through the application. This (anti-)pattern makes it impossible
to figure out whether two implementations of the same format or protocol are identical, and input
handling code can't be easily reused between applications. As a result, different applications often
disagree in the corner cases of a protocol, which can have fatal security consequences. For example,
Android has two parsers for ZIP archives involved in securely installing applications. First, a Java
program checks the signatures of files contained within an app archive and then another tool
extracts them to the filesystem. Because the two ZIP parsers disagree in multiple places, attackers
can modify a valid file so that the verifier will see the original contents, but attacker-controlled
files will be extracted to the filesystem, bypassing Android's code signing. Similar issues showed
up on iOS code signing~\cite{geohot-evasion} and SSL certificates~\cite{DBLP:conf/fc/KaminskyPS10}.


Instead of attempting to parse inputs by hand and failing, 
a promising approach is to specify
a precise grammar for the input data format, and automatically generate parsing code from that with tools like {\tt yacc}. As long
as the parser generator is bug-free, the application will be safe from many
input processing vulnerabilities. Grammars can also be re-used between
applications, further reducing effort and eliminating inconsistencies.


This approach is typical in compiler design and in other applications handling text-based inputs, but not
common for binary inputs. The Hammer framework~\cite{hammer-parser} and data description languages such as
PADS~\cite{Fisher:2005:PDL:1064978.1065046} have been developing generated parsers for binary protocols.

However, if you wanted to use existing tools to parse PDF
or ZIP files, you would soon find that they cannot handle the complicated---and therefore most
error-prone---aspects of such formats, so you'd still have to
hand-write the riskiest bits of code. For example, existing parser generators cannot conveniently represent size or offset fields, and more complex features, such as data compression or checksums, cannot be expressed at all.


Furthermore, some parser generators are cumbersome to use when parsing binary data for several
reasons. First, many parser generators don't produce convenient data structures, and instead call
but call \emph{semantic actions} that the developer has to write to build up a data structure that
the rest of the program can use. Therefore, the developer must
describe the format up to three times: in the grammar, in the data structure, and in the semantic
actions. Second, most parser generators only address parsing inputs, so the developer would have to manually
construct outputs. Some parser generators, such as 
Boost.Spirit, allow generating output, but require the developer to write another set of semantic actions. 
\section{Approach}

We address these challenges with Nail, a new parser generator for binary formats:


First, parser generators typically parse inputs into an abstract syntax
tree (AST) that corresponds to the grammar.  In order to produce a data
structure that the rest of the application code can easily process,
application developers must write explicit {\em semantic actions} that
update the application's internal representation of the data based on
each AST node.  
Writing these semantic actions requires the programmer to describe the
structure of the input three
times---once to describe the grammar, once to describe the internal data
structure, and once again in the semantic actions that translate the
grammar into the data structure---leading to potential bugs and
inconsistencies.
Furthermore, in a memory-unsafe language like C, these semantic
actions often involve error-prone manual memory allocation and
pointer manipulation.


Second,  applications often need to produce output in the same
format as their input---for example, applications might both
read and write files, or both receive and send network packets.
Most parser generators focus on just parsing an input, rather
than producing an output, thus requiring the programmer to manually
construct outputs, which is work-intensive and leads to
more code that could contain errors.
Some parser generators, such as
Boost.Spirit~\cite{boost-spirit},
allow reusing the grammar for generating output from the internal
representation.  However, those generators require yet another set of
semantic actions to be written, transforming the internal representation
back into an AST\@.

Third, many data formats contain redundancies, such as repeating
information in multiple structures.  Applications usually do not
explicitly check for consistency, and if different applications use
different instances of the same value, an attacker can craft an input that
causes inconsistencies~\cite{evaders6}.  Furthermore, security
vulnerabilities can occur when an application assumes two repetitions
of the same data to be consistent, such as allocating a buffer based on
the value of one size field and copying into that buffer based on the
value of another~\cite{python-bug:20078}.

Finally, real-world data formats, such as PNG or PDF, are hard to
represent with existing parser generators.  Those parsers cannot directly
deal with length or checksum fields, so the programmer has to
either write potentially unsafe code to deal with such features, or build
contrived grammar constructs, such as introducing one grammar rule for
each possible value of a length field.  Offset fields, which specify
the position at which some data structure is located, usually require
the programmer to manipulate a parser's internal state to re-position
its input.  More complicated transformations, such as handling compressed
data, cannot be represented at all.

\section{Proposed Evaluation}
To show that Nail is practical to handle real world data formats, we
will demonstrate grammars for several real-world file formats and
protocols that have had security vulnerabilities in their
hand-implemented parsers. We will then show how these grammars can be
used to implement applications and compare their performance and
complexity to hand-written implementations. So far, we have grammars
for DNS packets and ZIP files, which we use to develop
proof-of-concept applications that have comparable performance to
existing systems used in practice.

We will also show a grammar-based IPv4 and IPv6  stack showing acceptable performance and reduced complexity.
\section{Previous work}

This proposal and the thesis are based on previously published papers about Nail, including \cite{bangert:nail-langsec}, \cite{bangert:nail-osdi14} and \cite{bangert:nail-login}.



Generating parsers and generators from an executable specification is the
core concept of interface generators, such as CORBA~\cite{omg:corba},
XDR~\cite{RFC:1832}, and Protocol Buffers~\cite{varda2008}.  However,
interface generators do not allow the programmer to specify the byte-level
data format; instead, they define their own data encoding that is
specific to a particular interface generator.  For instance, XDR-based
protocols are incompatible with Protocol Buffers.  Moreover, this means
that interface generators cannot be used to interact with existing protocols
that were not defined using that interface generator in the first place.
As a result, interface generators cannot be used to parse or generate
widely used formats such as DNS or ZIP, which is a goal for Nail.

Closely related work has been done in the field of data description languages, for example
PacketTypes~\cite{mccann2000packet} and DataScript~\cite{back2002datascript}; a
broader overview of data description languages can be found in Fisher et
al~\cite{Fisher:700DDL}.
PacketTypes implements a C-like structure model enhanced with
length fields and constraints, but works only as a parser, and
not as an output generator.
DataScript adds output generation and built-in support for offset fields.
A particularly sophisticated data description language, PADS~\cite{Fisher:2005:PDL:1064978.1065046},
which is targeted more towards offline analysis, even features built-in support for XML and
automatic grammar inference.
However, these systems cannot easily handle complicated encodings such as compressed data, which are
supported by Nail's stream transforms. While sophisticated languages like PADS allow for handling
particular variations of offset fields, compressed data, or even XML entities, each of these
features has to be implemented in the data description language and all associated tools. Nail's
transformations keep the core language small, while enabling the wide range of
features real-world protocols require. 

Recently, the Hammer project~\cite{hammer-parser} introduced a security-focused parser framework for
binary protocols.
Hammer implements grammars as language-integrated parser combinators, an approach popularized by
Parsec for Haskell~\cite{LeijenMeijer:parsec}. The parser combinator style (to our knowledge, first
described by Burge~\cite{burge1975recursive}) is a natural way of concisely expressing top-down
grammars~\cite{Danielsson:2010:TPC:1863543.1863585}
by composing them from one or multiple sub-parsers.\footnote{For more
background on the history of expressing grammars, see Bryan Ford's
masters thesis~\cite{ford2002packrat}, which also describes the
default parsing algorithm used by Hammer.}
Hammer then constructs a tree of function pointers which can be invoked to parse a given input into
an AST\@.

\begin{singlespace}
\bibliography{n-str,main,locasto,langsec,n,n-conf}

\end{singlespace}

\end{document}




