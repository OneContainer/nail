\section{Introduction}

Code that handles untrusted inputs, such as processing network
data or parsing a file, is error-prone and is often exploited by
attackers.  This is in part because attackers have precise control
over the inputs to that code, and can craft inputs that trigger
subtle corner cases in input processing.  For example, the libpng
library has had 24 remotely exploitable vulnerabilities from 2007 to
2013,\footnote{\url{http://www.cvedetails.com/vulnerability-list/vendor_id-7294/Libpng.html}}
and Adobe's PDF and Flash viewers have been notoriously
plagued by input processing vulnerabilities.  Even relatively
simple formats, such as those used by the zlib compression
library, have had input processing vulnerabilities in the
past.\footnote{\url{http://www.cvedetails.com/vulnerability-list/vendor_id-72/product_id-1820/GNU-Zlib.html}}

A promising approach to avoid such vulnerabilities is to specify
a precise grammar for the input data format, and to use a parser
generator, such as {\tt lex} and {\tt yacc}, to synthesise the input
processing code.  Developers that use a parser generator do not need
to write error-prone input processing code on their own, and as long
as the parser generator is bug-free, the application will be safe from
input processing vulnerabilities.  Unfortunately, applying this approach
in practice, using state-of-the-art parser generators, still requires
too much manual programmer effort, making it error-prone, as we describe
next.

First, parser generators typically parse inputs into an abstract syntax
tree (AST) that corresponds to the grammar.  In order to produce a data
structure that the rest of the application code can easily process,
application developers must write explicit {\em semantic actions} that
update the application's internal representation of the data based on
each AST node.  Writing these semantic actions by hand is error-prone,
much like other input processing code, and mistakes can result in memory
corruption bugs or misinterpreted inputs.  Writing these semantic actions
also requires the programmer to describe the structure of the input three
times---once to describe the grammar, once to describe the internal data
structure, and once again in the semantic actions that translate the
grammar into the data structure---leading to another potential source
of bugs and inconsistencies.

Second,  applications often need to produce output in the same
format as their input---for example, applications might both
read and write files, or both receive and send network packets.
Most parser generators just focus on parsing an input, rather
than producing an output, thus requiring the programmer to manually
construct outputs, which is error-prone.  Some parser generators, such as
Boost.Spirit~\cite{boost-spirit},
allow reusing the grammar for generating output from the internal
representation.  However, those generators require yet another set of
semantic actions to be written, transforming the internal representation
into an AST\@.


Third, data formats, especially binary formats such as PNG or PDF, can have \emph{structural dependencies}, such as offset, length, and checksum fields
that are hard to express in state-of-the-art grammar languages. Often the programmer is required to
manually write control code, such as re-positioning the parser's input stream, looping over the
invocation of a sub-parser, or computing a
checksum over raw input bytes.  Besides leaving much room for errors with offset arithmetic, such
code is usually not reusable when generating output.


This paper presents the initial design and implementation of Nail, a parser
generator that greatly reduces the programmer effort required to use a
grammar-based parser. Nail addresses the above three challenges with several key
ideas.

First, Nail reduces the expressiveness of its grammar language by removing
semantic actions. Existing parser generators allow arbitrary computation to
transform between the AST and the parser output. Instead, Nail derives the
structure of its output automatically from the grammar, forcing the programmer
to clearly separate syntactic validation and semantic processing.

Second, this well-defined internal representation allows Nail to invert the
parser and generate output. However, Nail allows \emph{constants} in the
external format to have multiple representations if they should not affect the
semantics of the data. For example, in a text protocol, the amount of
white-space separating tokens should not affect the meaning of the data and
consequently Nail does not expose it to the application. As long as
constants are only used for their intended purpose of representing syntax-only
features, the generated output will have the same semantics as the parsed input.

Third, Nail provides support for structural dependencies, transparently handling
offset and length fields. By hiding the offset and length values from the
programmer, Nail ensures they remain consistent even if the data is changed by
the application. 


%  XXX todo.  Bijections.  Reduced
% expressiveness.  Constraints.  External interaction (checksum, ports,
% etc).

We have implemented an initial prototype of Nail for C\@.  Our experience
so far suggests that it is a promising approach: we were able to construct
a succinct grammar for DNS packets, and write a small DNS server that
uses Nail for all packet input and output, and operates purely on
Nail-generated data structures, with no need for any semantic actions.

The rest of this paper is organized as follows.  \S\ref{s:relwk}
puts Nail in the context of related work.  \S\ref{s:design} describes
Nail's design.  \S\ref{s:impl} discusses our initial implementation
of Nail.  \S\ref{s:eval} provides some initial evaluation results.
\S\ref{s:future} suggests several directions for future work, and
\S\ref{s:concl} concludes.

