\section{Design}
\label{s:design}

\subsection{Overview}

To integrate the data format and the internal representation, Nail
provides a rich set of combinators that not
only describe the grammar of the external protocol, but also induce an internal
model.

A central design decision of Nail is that there is a \emph{semantic bijection} between
the model exposed to the programmer and the byte-level input and output, up to
syntactic equivalence for unambiguous grammars. More precisely, the parser is
the generator's inverse, so parsing the generator's output will yield the
generator's input, but generating the parsers output does not necessarily yield
the parsers input. To understand why this makes sense, consider a grammar for a
text language that tolerates white space, or a binary protocol that tolerates
arbitrarily long padding.\footnote{Say, the physical layer of most communication
  protocols is a possibly infinite sequence of symbols that are syntactically
  nil followed by a pre-determined synchronization sequence and the actual
  contents of the transmission.} In that case, the program semantics should be
independent of the number of padding elements in the input, and Nail therefore
does not expose that information to the programmer. We call such discarded
fields \emph{constants}. Currently, Nail always makes a default choice when
there are multiple options to express a constant, however Nail could be extended
to allow a grammar-specific plug-in to make these choices, say for faster
alignment, streaming applications when data is not ready, or visual appearance in
human-readable protocols. Similarly, Nail does not preserve the layout of
objects referred to by their offsets. If the grammar contains no constants and
offset fields, there is a proper isomorphism between model and protocol.


Nail evaluates the combinators and produces:

\begin{itemize}
\item type declarations for the internal model,
\item the \textit{parser}, a function to parse a sequence of bytes into an
instance of the model, and
\item the \textit{generator}, a function to create a
sequence of bytes from an instance of the model.
\end{itemize}

\noindent
In the rest of this section, we present Nail's combinators.

\begin{figure}[tb]
\begin{tabular}{ll}
\toprule
\bf Syntax & \bf Semantics \\
\midrule
\texttt{int32} & 32-bit signed integer \\
\multirow{2}{*}{\texttt{uint4}}
  & 4-bit unsigned integer, \\
  & returned as an 8-bit value \\
\multirow{2}{*}{\texttt{uint8 | 1..3}}
  & 8-bit unsigned integer, \\
  & $1\leq x \leq 3$ \\
\multirow{2}{*}{\texttt{uint16 | ..512}}
  & unsigned 16-bit integer, \\
  & $x \leq 512$ \\
\multirow{2}{*}{\texttt{int32 | [1,255,512]}}
  & signed 32 bit integer, \\
  & $x \in \{ 1, 255, 512 \}$ \\
\bottomrule
\end{tabular}
\caption{Example Nail grammars for integer values.}
\label{fig:range}
\end{figure}

\paragraph{Fundamental parsers.}
The elementary parsers of Nail are the same as those of Hammer, signed
and unsigned integers with arbitrary lengths up to 64 bits.  Note that is
possible to define parsers for sub-byte lengths, e.g. to parse the 4-bit
data offset within the TCP header; in Nail's syntax.  Integer parsers
return their value in the smallest appropriately sized machine integer
type; e.g., a 24-bit integer is stored in a 32-bit wide variable.

Integer parsers can be constrained to fall either within an (inclusive)
range of values or be an element of a set of allowed values.
Figure~\ref{fig:range} shows several examples.
Invalid values or prematurely reaching the end of input will raise an
error when parsing input or generating output, and abort the procedure.


\paragraph{Sequence.}

Nail's fundamental concept is the structure combinator. It contains a list of
named parsers and unnamed constant parsers. The parser invokes each field 
in sequence and returns a structure containing the result of each parser.
For example, Figure~\ref{fig:dns-struct} shows the structure combinator
from a part of the grammar for DNS packets, along with the data model
corresponding to it.

\begin{figure}[tb]

\begin{minipage}{0.45\columnwidth}
\begin{verbatim}
header = {
  id uint16
  qr uint1
  opcode uint4
  aa uint1 
  tc uint1
  rd uint1
  ra uint1
  uint3 = 0
  rcode uint4
}
\end{verbatim} 
\end{minipage}
~
\begin{minipage}{0.45\columnwidth}
\begin{verbatim}
struct header {
  uint16_t id;
  uint8_t qr;
  uint8_t opcode;
  uint8_t aa;
  uint8_t tc;
  uint8_t rd;
  uint8_t ra;
  uint8_t rcode;
};
\end{verbatim} 
\end{minipage}

\caption{Nail grammar (left) and data model (right) for a part of the
grammar for DNS packets.  The \texttt{uint3 = 0} grammar represents
3 bits of padding (filled with zeroes).}
\label{fig:dns-struct}
\end{figure}


\paragraph{Repetition.}

The \emph{many} combinator takes a parser and applies it repeatedly
until it fails, returning an array of the inner parsers results. The
\emph{sepBy} combinator
additionally takes a constant parser, which it applies in between parsing
two values, but not before parsing the first value or after parsing the
last.
For example, \texttt{many uint8} represents an array of 8-bit unsigned
integers, and \texttt{sepBy uint8=',' (many uint8 | '0'..'9')} recognizes
comma-separated lists of decimal numbers.


\paragraph{Semantic choice.}

We extend a parsing expression grammar's ordered choice combinator with
a tag for each choice. The parser
attempts to parse each option in the order they are specified in the field and
stores the result in a tagged union. If an option fails, the parser backtracks
to the beginning of the choice combinator's input. Care must be taken that the
choices do not overlap, because Nail always picks the first successfully
parsed choice. If two options overlap, generated output for the latter option is
not necessarily understood identically by the parser. However, actual data
formats are usually not ambiguous in this sense.
Figure~\ref{fig:choice} demonstrates a simple choice combinator.

\begin{figure}[tb]
\begin{verbatim}
choose {
  A = uint8 | 1..8
  B = uint16 | ..256
}
\end{verbatim}
\caption{A simple choice combinator that parses either an 8-bit unsigned
integer with a value between 1 and 8 (option A), or a 16-bit unsigned
integer with a value of at most 256 (option B).}
\label{fig:choice}
\end{figure}

\subsection{Constant parsers}

Nail also features \emph{constant parsers}, which do not affect the internal
representation. Constant parsers can appear instead of structure fields and as
separators in the sepBy combinator.

The simplest constant parser is an integer or array of integers with fixed
value; for example, \texttt{uint8=0}, or \texttt{many uint8=[1,2]}. A convenience notation
for strings is also supported: \texttt{many uint8 = "foo"}.

\paragraph{Wrap combinator.} When implementing real protocols with Nail, we often found that
structures that consist of many constant parsers and only one named field. This pattern is
common in binary protocols which use fixed headers to denote the type of data
structure to be parsed.  In order to keep the internal representation cleaner,
we introduced the wrap combinator, which takes a sequence of parsers containing
exactly one non-constant parser. The parser and generator act as though the wrap
combinator was a sequence of parsers, but the data model does not wrap the
single value in another structure, making the application-visible representation
(and thus application code) more concise.

For example, \texttt{<uint8='"'; many int8 | 'a'..'z'; uint8='"'>} parses a quoted
lowercase word into an array, excluding the quotation marks.

\paragraph{Constant combinators.}
In some protocols, there might be many ways to represent the same constant field
and there is no semantic difference between the different syntactic representations.
To support this pattern, Nail allows developers to use choice and repetition
combinators together with constant fields, such as
\texttt{many (uint8=' ')} (representing any number of space characters), or
\texttt{|| uint8 = 0x90 || uint16 = 0x1F0F} (parsing two of the many representations of NOP on the
x86 architecture). Note that constant may have varying lengths.
This is particularly useful for handling padding or whitespace.

As discussed above, choosing to use these combinators on constant parsers
removes the bijection between byte-strings and our data model, as there are
multiple byte-strings that correspond to the same internal data structure and
the generator has to choose one of these representations. As such, constant
combinators are the generator dual of ambiguous choice combinators in the
parser, because they lead to ambiguities in the generator.


\subsection{Dependent fields}
Another problem for parser generators is that binary protocols often contain
length and offset fields. Conventional parsing algorithms can, in principle,
deal with
bounded offset fields: a finite automaton can count a bounded integer, and we
can feed the (finite) input multiple times to the finite automaton. However,
this implementation is both time-inefficient (it feeds many bytes into the
automaton that will just be skipped) and very cumbersome to express with
current parser generators. Therefore, if languages with offset fields need
to be parsed
with parser generators, the only currently practical way is to add ad-hoc hacks
such as changing the input pointer of the generated parser on the fly, as part
of the code in the semantic action for the offset field.

Nail will properly support both offset and length fields and much of the
following discussion applies to both, although the current prototype only
implements lengths, which we will focus on.

We call length or offset fields \textit{dependent fields}, because during
parsing, another parser depends on them, and while generating output, their
value depends on some other structure in the data model. Dependency fields
appear in a structure combinator as would any other integer field, but
their name begins with an \texttt{@} sign.  A dependency field has to
appear in the grammar before it can be used.

Dependency fields are not exposed in the data model, but instead are
transparently computed.  This frees the developer from checking that
these fields are correct (for input) or having to keep their values in
sync with the rest of the data structure (for output).

\paragraph{Length fields.}

The length combinator, \texttt{n_of}, takes a dependency field $n$
and a parser, evaluates the parser exactly $n$ times (i.e., setting
the number of iterations to be the $n$ field's value), and returns
an array of the parser's values. When generating output, it emits
the array and writes its length to the dependency field.

\paragraph{Offsets.}

The \texttt{offset} combinator takes a dependency field and a parser. It
moves the parser to the position specified in the offset field and
invokes the inner parser, and then moves the input back to its original
position.  While generating output, all structures referred to by offset
are generated after the main structure and the dependent offset fields
are patched up.

\paragraph{Checksums.}

In many data formats, some values depend on external representation,
such as checksums and cryptographic signatures.  While it would be
possible to extend our constraint language to be powerful enough to
support such constructs, we would essentially be building a separate,
Turing-complete language that has all the same pitfalls existing programs
have.  Therefore, we intend to allow the programmer to carefully escape
Nail's programming model and write a function that takes a pointer to
the dependent value and a range of bytes, using the \texttt{raw_depend}
combinator.

For example, we imagine the following grammar could be used to represent
a sequence of bytes \texttt{data} followed by its CRC32 checksum:

\begin{verbatim}
data many uint8; @checksum uint32
raw_depend @checksum data crc32
\end{verbatim}

\noindent
where \texttt{crc32} is a function supplied by the application, with
the following signature:

\begin{verbatim}
bool crc32(uint32_t *out, uint8_t *in);
\end{verbatim}

Because this feature compromises Nail's security guarantees, it should
only be used in limited circumstances and with carefully prepared checksum
functions.  This feature is not implemented in the current prototype.

