\section{Design}
\label{s:design}

\paragraph{Overview}
To alleviate these problems, Nail provides a richer set of combinators that induce the structure of
an internal model, while also describing the grammar of the external protocol. Nail takes the
combinators and produces type declarations for an internal model, the \textit{parser}, a function to parse a sequence of
bytes into an instance of the model and the \textit{generator}, a function to create a sequence of
bytes from an instance of the model.

A central design decision of Nail is that there is a  bijection between the model exposed to the programmer and the bytes
input and output up to syntactic equivalence. More precisely, the parser is the generators inverse,
so parsing the generators output will yield the generators input, but generating the parsers output
does not necessarily yield the parsers input. To understand why this makes sense, consider a grammar
for a text language that tolerates white space or a binary protocol that tolerates arbitrarily long
padding\footnote{Say, the physical layer of most communication protocols is a possibly infinite
  sequence of symbols that are syntactically nil followed by a pre-determined synchronisation
  sequence and the actual contents of the transmission.}. However, if some features of Nail are
avoided, there is a proper isomorphism between model and protocol.

In the following, we present the combinators in Nail grouped by the combinators of conventional Parser Expression Grammars they replace.
\paragraph{Fundamental parsers}
The elementary parsers of Nail are the same as those of Hammer, signed and unsigned integers with
arbitrary lengths up to 64 bits. Note that is possible to define parsers for sub-byte lengths, e.g.
to parse the 4-bit data offset within the TCP header. Integer parsers return their value in the
smallest appropriately sized machine integer type, e.g. a 24 bit integer is stored in a 32-bit wide variable.\\
Examples: \texttt{uint4}, \texttt{int32}

Integer parsers can be constrained to fall either within an (inclusive) range of values or be an
element of a set of allowed values. \\
Examples: \texttt{uint8 | 1..3} (accepts an 8 bit unsigned
integer between 1 and 3 inclusive) ,\\
 \texttt{uint16| ..512} (an unsigned 16-bit integer less or
equal than 512),\\
\texttt{int32| [1,255,512]} (a signed 32 bit integer with value 1, 255 or 512). 

\paragraph{Constant parser}
Nail also has constant parsers, which do not return a value. The simplest constant parser is an
integer with fixed value, e.g. \texttt{uint8=0}. Constant parsers succeed if the inner integer
parser returns the given value and fail otherwise. 


\paragraph{Sequence: Structure}
Nails fundamental concept is the structure combinator. It contains a list of named parsers and
unnamed constant parsers. 
When parsing, it parses each parser in sequence and returns a structure containing the result of
each parser.

\begin{tabular}{ll}
\textbf{Nail} & \textbf{Data Model}\\
\begin{minipage}{2in}
\begin{verbatim}
header = {
 id uint16
 qr uint1
 opcode uint4
 aa uint1 
 tc uint1
 rd uint1
 ra uint1
 uint3 = 0
 rcode uint4
}
\end{verbatim} 
\end{minipage}
 & 
\begin{minipage}{2in}
\begin{verbatim}
struct header{
    uint16_t id;
    uint8_t qr;
    uint8_t opcode;
    uint8_t aa;
    uint8_t tc;
    uint8_t rd;
    uint8_t ra;
    uint8_t rcode;
};
\end{verbatim} 
\end{minipage} 
\\
\end{tabular}
\paragraph{Repetition: Many,SepBy}
The many combinator takes a parser and applies it repeatedly until it fails, returning an array of
the inner parsers results. SepBy additionally takes a constant parser, which it applies in between
parsing two values,but not before parsing the first value or after parsing the last.\\
Examples \texttt{many uint8}, \texttt{sepBy uint8=',' (many uint8| '0'..'9')} (which recognises
comma-separated lists of numbers).

The many combinator also works on constant parsers, although as described above it can make the
generator not bijective.\footnote{By default, the generator will emit exactly one iteration of the
  constant parser in its output.} Example: \texttt{many uint8= "test"}


\paragraph{Semantic choice: choose}
We extend PEG's ordered choice combinator, which picks the first choice that succeeds, with a tag
for each choice. The result is saved in a tagged union. 

\begin{verbatim}
choose {
   A = uint8 | 1.. 8
   B = uint16 | ..256
}
\end{verbatim}
\paragraph{Wrap}

The wrap combinator parses one or more constant parser, then an inner parser whose value it returns
and then again parses one or more constant parsers.\\ Example: 
\verb+<uint8='"'; many8 uint8|'a'..'z'; uint8='"'  >+

\subsection{Offset fields}
Another problem for parser generators is that binary protocols often contain length and offset
fields. While conventional parsing algorithms can deal with bounded offset fields - a finite
automaton can count a bounded integer, and we can feed the (finite) input multiple times to the
finite automaton. However, this implementation is both time-inefficient - why feed many bytes into
the automaton that will just be skipped - and very cumbersome to express with current parser
generators. Therefore, if languages with offset fields are parsed with parser generators, the only
currently feasible way is to add 'ad-hoc' hacks such as changing the input pointer of the generated
parser on the fly from semantic actions. 

Nail will properly support both offset and length fields and much of the following discussion
applies to both , although the current prototype only implements lengths, which we will focus on. 

\paragraph{Dependency fields}
We call length or offset fields \textit{dependency fields}, because during parsing, another parser
depends on them, and while generating output, their value depends on some other structure in the
data model.
Dependency fields appear in a structure combinator as would any other field, but their name begins
with an '@' sign and they have to return an integer value. Dependency fields are not directly
represented in the internal model. 

\paragraph{Length fields: n_of}
The length combinator takes a dependency field and a parser, evaluating the parser as often as the
value of the dependency field, returning an array of the values. When generating output, it emits
the array and writes its length to the dependency field. 

\subsection{Example Grammars}
To further familiarise the reader with Nail, we provide two example grammars. 
\paragraph{UTF-16}
This grammar recognises valid UTF-16 strings and exposes an array of code points. 

\begin{tabular}{ll}
\textbf{Nail} & \textbf{Data Model}\\
\begin{minipage}{3in}
\begin{verbatim}
utfstring = many choose {
         SUPP= {
               lead uint16 | 55296..56319
                          // 0xD800..0xDBFF
               trail uint16 | 56320..57343
                          // 0xDC00..0xDFFF
               }
         BASIC = uint16 | !55296..57343
       }
\end{verbatim}
\end{minipage}
 & 
\begin{minipage}{2in}
\begin{verbatim}
struct utfstring {
 struct {
   enum  {SUPP,BASIC} N_type;
   union {
    struct {
     uint16_t lead;
     uint16_t trail;
    } SUPP;
    uint16_t BASIC;
   };
 }*elem;
 size_t count;
};
\end{verbatim} 
\end{minipage} 
\\
\end{tabular}

\paragraph{DNS packet}
This grammar recognises DNS packets without label compression, as per RFC1035 
\begin{verbatim}
labels = <many {@length  uint8 | 1..255 
                label n_of @length uint8 }
             uint8 = 0>
question= {   labels labels
              qtype uint16 | 1..16 
              qclass uint16 | [1,255]
               }
answer  = {
             labels labels
             rtype uint16 | 1..16
             class uint16 | [1]
             ttl uint32
             @rlength uint16 
             rdata n_of @rlength uint8 
             }
dnspacket = {id uint16
              qr uint1
              opcode uint4
              aa uint1 
              tc uint1
              rd uint1
              ra uint1
              uint3 = 0
              rcode uint4
              @questioncount uint16
              @answercount uint16
              authoritycount uint16   // Ignored
              additionalcount uint16  //Ignored       
              questions n_of @questioncount question         
              responses n_of @answercount answer
            }
\end{verbatim}


