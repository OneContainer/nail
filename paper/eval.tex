\section{Evaluation}

\label{s:eval}
In the following section we will present two short Nail grammars for UTF-16
encoded strings and for a subset of DNS packets sent to and from an
authoritative name server. Then we present a functioning toy DNS server that we
have implemented to demonstrate the feasibility of using Nail's internal
representation.

As Nail is work in progress, many parts of the implementation, syntax and design
are not complete yet and we do not yet have meaningful performance or security metrics.

\subsection{Example Grammars}
\paragraph{UTF-16.}
This grammar recognises valid UTF-16 strings and exposes an array of code
points.
\hspace{-2em}\begin{tabular}{ll}
  \textbf{Nail} & \textbf{Data Model}\\
  \begin{minipage}{1.3in}
\begin{verbatim}
utf = many choose {
 SUPP= {
  lead uint16  |
     0xD800..0xDBFF
  trail uint16 | 
     0xDC00..0xDFFF
 }
 BASIC = uint16 |
    !0xD800..0xDFFF
}
\end{verbatim}
  \end{minipage}
  & 
  \begin{minipage}{2in}
\begin{verbatim}
struct utfstring {
 struct {
   enum  {SUPP,BASIC} 
         N_type;
   union {
    struct {
     uint16_t lead;
     uint16_t trail;
    } SUPP;
    uint16_t BASIC;
   };
 }*elem;
 size_t count;
};
\end{verbatim} 
  \end{minipage}
  \\
\end{tabular}

\paragraph{DNS packet}
This grammar recognises DNS packets without label compression, as per RFC1035.
\begin{verbatim}
labels = <many {@length  uint8 | 1..255 
                label n_of @length uint8 }
             uint8 = 0>
question= {   labels labels
              qtype uint16 | 1..16 
              qclass uint16 | [1,255]
          }
answer  = {   labels labels
              rtype uint16 | 1..16
              class uint16 | [1]
              ttl uint32
              @rlength uint16 
              rdata n_of @rlength uint8 
          }
dnspacket = { id uint16
              qr uint1
              opcode uint4
              aa uint1 
              tc uint1
              rd uint1
              ra uint1
              uint3 = 0
              rcode uint4
              @qc uint16
              @ac uint16
              uint16 = 0   // authority 
              uint16 = 0 // additional
// We don't support authority or 
// additional sections in the prototype
              questions n_of @qc question         
              responses n_of @ac answer
            }
\end{verbatim}
  
Furthermore, our GitHub
repository contains other Nail grammars, such as a TAP network stack that processes Ethernet, ARP,
ICMP, IP and UDP and the grammar for Nail itself. 

\subsection{Applications}

In order to test the Nail framework, we cloned the test DNS server from the Hammer distribution to
Nail. Hammer ships with a toy DNS server written in 683 lines of code, excluding the hammer
framework itself that responds to any valid DNS query with a CNAME record to the domain
``spargelze.it''.  Most of this code is taken up with custom validators, semantic actions and data
structure definitions, with only 52 lines of code defining the grammar with Hammer's combinators.

Our DNS server measures 148 lines of C and 48 lines of Nail grammar, and supports a custom zone file format
with A,NS,MX and CNAME records. The same grammar is used together with 98 lines of C to implement a
functional toy clone of the host command line tool. However, because our grammar does not yet
support DNS label compression, the latter tool will occasionally reject valid real world DNS
responses. Both clients have functional anti-spoofing measures.

It is hard to compare our efforts to a real world DNS server, as we have less functionality, in
particular for DNS compression and additional hint records real-world DNS servers send. 
However, the closest in functionality and intent is Dan Bernstein's djbdns\footnote{\url{http://cr.yp.to/djbdns.html}}, which aims
to be a minimalist, highly secure DNS server. The latest release of DJBDNS, however, including
various support tools is about ten thousand lines of C as measured by SLOCCOUNT. We are confident
that it is possible to a feature-par version with Nail that is an order of magnitude smaller and
intend to do so.


